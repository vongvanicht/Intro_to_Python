
% Default to the notebook output style

    


% Inherit from the specified cell style.




    
\documentclass[11pt]{article}

    
    
    \usepackage[T1]{fontenc}
    % Nicer default font (+ math font) than Computer Modern for most use cases
    \usepackage{mathpazo}

    % Basic figure setup, for now with no caption control since it's done
    % automatically by Pandoc (which extracts ![](path) syntax from Markdown).
    \usepackage{graphicx}
    % We will generate all images so they have a width \maxwidth. This means
    % that they will get their normal width if they fit onto the page, but
    % are scaled down if they would overflow the margins.
    \makeatletter
    \def\maxwidth{\ifdim\Gin@nat@width>\linewidth\linewidth
    \else\Gin@nat@width\fi}
    \makeatother
    \let\Oldincludegraphics\includegraphics
    % Set max figure width to be 80% of text width, for now hardcoded.
    \renewcommand{\includegraphics}[1]{\Oldincludegraphics[width=.8\maxwidth]{#1}}
    % Ensure that by default, figures have no caption (until we provide a
    % proper Figure object with a Caption API and a way to capture that
    % in the conversion process - todo).
    \usepackage{caption}
    \DeclareCaptionLabelFormat{nolabel}{}
    \captionsetup{labelformat=nolabel}

    \usepackage{adjustbox} % Used to constrain images to a maximum size 
    \usepackage{xcolor} % Allow colors to be defined
    \usepackage{enumerate} % Needed for markdown enumerations to work
    \usepackage{geometry} % Used to adjust the document margins
    \usepackage{amsmath} % Equations
    \usepackage{amssymb} % Equations
    \usepackage{textcomp} % defines textquotesingle
    % Hack from http://tex.stackexchange.com/a/47451/13684:
    \AtBeginDocument{%
        \def\PYZsq{\textquotesingle}% Upright quotes in Pygmentized code
    }
    \usepackage{upquote} % Upright quotes for verbatim code
    \usepackage{eurosym} % defines \euro
    \usepackage[mathletters]{ucs} % Extended unicode (utf-8) support
    \usepackage[utf8x]{inputenc} % Allow utf-8 characters in the tex document
    \usepackage{fancyvrb} % verbatim replacement that allows latex
    \usepackage{grffile} % extends the file name processing of package graphics 
                         % to support a larger range 
    % The hyperref package gives us a pdf with properly built
    % internal navigation ('pdf bookmarks' for the table of contents,
    % internal cross-reference links, web links for URLs, etc.)
    \usepackage{hyperref}
    \usepackage{longtable} % longtable support required by pandoc >1.10
    \usepackage{booktabs}  % table support for pandoc > 1.12.2
    \usepackage[inline]{enumitem} % IRkernel/repr support (it uses the enumerate* environment)
    \usepackage[normalem]{ulem} % ulem is needed to support strikethroughs (\sout)
                                % normalem makes italics be italics, not underlines
    

    
    
    % Colors for the hyperref package
    \definecolor{urlcolor}{rgb}{0,.145,.698}
    \definecolor{linkcolor}{rgb}{.71,0.21,0.01}
    \definecolor{citecolor}{rgb}{.12,.54,.11}

    % ANSI colors
    \definecolor{ansi-black}{HTML}{3E424D}
    \definecolor{ansi-black-intense}{HTML}{282C36}
    \definecolor{ansi-red}{HTML}{E75C58}
    \definecolor{ansi-red-intense}{HTML}{B22B31}
    \definecolor{ansi-green}{HTML}{00A250}
    \definecolor{ansi-green-intense}{HTML}{007427}
    \definecolor{ansi-yellow}{HTML}{DDB62B}
    \definecolor{ansi-yellow-intense}{HTML}{B27D12}
    \definecolor{ansi-blue}{HTML}{208FFB}
    \definecolor{ansi-blue-intense}{HTML}{0065CA}
    \definecolor{ansi-magenta}{HTML}{D160C4}
    \definecolor{ansi-magenta-intense}{HTML}{A03196}
    \definecolor{ansi-cyan}{HTML}{60C6C8}
    \definecolor{ansi-cyan-intense}{HTML}{258F8F}
    \definecolor{ansi-white}{HTML}{C5C1B4}
    \definecolor{ansi-white-intense}{HTML}{A1A6B2}

    % commands and environments needed by pandoc snippets
    % extracted from the output of `pandoc -s`
    \providecommand{\tightlist}{%
      \setlength{\itemsep}{0pt}\setlength{\parskip}{0pt}}
    \DefineVerbatimEnvironment{Highlighting}{Verbatim}{commandchars=\\\{\}}
    % Add ',fontsize=\small' for more characters per line
    \newenvironment{Shaded}{}{}
    \newcommand{\KeywordTok}[1]{\textcolor[rgb]{0.00,0.44,0.13}{\textbf{{#1}}}}
    \newcommand{\DataTypeTok}[1]{\textcolor[rgb]{0.56,0.13,0.00}{{#1}}}
    \newcommand{\DecValTok}[1]{\textcolor[rgb]{0.25,0.63,0.44}{{#1}}}
    \newcommand{\BaseNTok}[1]{\textcolor[rgb]{0.25,0.63,0.44}{{#1}}}
    \newcommand{\FloatTok}[1]{\textcolor[rgb]{0.25,0.63,0.44}{{#1}}}
    \newcommand{\CharTok}[1]{\textcolor[rgb]{0.25,0.44,0.63}{{#1}}}
    \newcommand{\StringTok}[1]{\textcolor[rgb]{0.25,0.44,0.63}{{#1}}}
    \newcommand{\CommentTok}[1]{\textcolor[rgb]{0.38,0.63,0.69}{\textit{{#1}}}}
    \newcommand{\OtherTok}[1]{\textcolor[rgb]{0.00,0.44,0.13}{{#1}}}
    \newcommand{\AlertTok}[1]{\textcolor[rgb]{1.00,0.00,0.00}{\textbf{{#1}}}}
    \newcommand{\FunctionTok}[1]{\textcolor[rgb]{0.02,0.16,0.49}{{#1}}}
    \newcommand{\RegionMarkerTok}[1]{{#1}}
    \newcommand{\ErrorTok}[1]{\textcolor[rgb]{1.00,0.00,0.00}{\textbf{{#1}}}}
    \newcommand{\NormalTok}[1]{{#1}}
    
    % Additional commands for more recent versions of Pandoc
    \newcommand{\ConstantTok}[1]{\textcolor[rgb]{0.53,0.00,0.00}{{#1}}}
    \newcommand{\SpecialCharTok}[1]{\textcolor[rgb]{0.25,0.44,0.63}{{#1}}}
    \newcommand{\VerbatimStringTok}[1]{\textcolor[rgb]{0.25,0.44,0.63}{{#1}}}
    \newcommand{\SpecialStringTok}[1]{\textcolor[rgb]{0.73,0.40,0.53}{{#1}}}
    \newcommand{\ImportTok}[1]{{#1}}
    \newcommand{\DocumentationTok}[1]{\textcolor[rgb]{0.73,0.13,0.13}{\textit{{#1}}}}
    \newcommand{\AnnotationTok}[1]{\textcolor[rgb]{0.38,0.63,0.69}{\textbf{\textit{{#1}}}}}
    \newcommand{\CommentVarTok}[1]{\textcolor[rgb]{0.38,0.63,0.69}{\textbf{\textit{{#1}}}}}
    \newcommand{\VariableTok}[1]{\textcolor[rgb]{0.10,0.09,0.49}{{#1}}}
    \newcommand{\ControlFlowTok}[1]{\textcolor[rgb]{0.00,0.44,0.13}{\textbf{{#1}}}}
    \newcommand{\OperatorTok}[1]{\textcolor[rgb]{0.40,0.40,0.40}{{#1}}}
    \newcommand{\BuiltInTok}[1]{{#1}}
    \newcommand{\ExtensionTok}[1]{{#1}}
    \newcommand{\PreprocessorTok}[1]{\textcolor[rgb]{0.74,0.48,0.00}{{#1}}}
    \newcommand{\AttributeTok}[1]{\textcolor[rgb]{0.49,0.56,0.16}{{#1}}}
    \newcommand{\InformationTok}[1]{\textcolor[rgb]{0.38,0.63,0.69}{\textbf{\textit{{#1}}}}}
    \newcommand{\WarningTok}[1]{\textcolor[rgb]{0.38,0.63,0.69}{\textbf{\textit{{#1}}}}}
    
    
    % Define a nice break command that doesn't care if a line doesn't already
    % exist.
    \def\br{\hspace*{\fill} \\* }
    % Math Jax compatability definitions
    \def\gt{>}
    \def\lt{<}
    % Document parameters
    \title{09\_Simulation-Copy1}
    
    
    

    % Pygments definitions
    
\makeatletter
\def\PY@reset{\let\PY@it=\relax \let\PY@bf=\relax%
    \let\PY@ul=\relax \let\PY@tc=\relax%
    \let\PY@bc=\relax \let\PY@ff=\relax}
\def\PY@tok#1{\csname PY@tok@#1\endcsname}
\def\PY@toks#1+{\ifx\relax#1\empty\else%
    \PY@tok{#1}\expandafter\PY@toks\fi}
\def\PY@do#1{\PY@bc{\PY@tc{\PY@ul{%
    \PY@it{\PY@bf{\PY@ff{#1}}}}}}}
\def\PY#1#2{\PY@reset\PY@toks#1+\relax+\PY@do{#2}}

\expandafter\def\csname PY@tok@w\endcsname{\def\PY@tc##1{\textcolor[rgb]{0.73,0.73,0.73}{##1}}}
\expandafter\def\csname PY@tok@c\endcsname{\let\PY@it=\textit\def\PY@tc##1{\textcolor[rgb]{0.25,0.50,0.50}{##1}}}
\expandafter\def\csname PY@tok@cp\endcsname{\def\PY@tc##1{\textcolor[rgb]{0.74,0.48,0.00}{##1}}}
\expandafter\def\csname PY@tok@k\endcsname{\let\PY@bf=\textbf\def\PY@tc##1{\textcolor[rgb]{0.00,0.50,0.00}{##1}}}
\expandafter\def\csname PY@tok@kp\endcsname{\def\PY@tc##1{\textcolor[rgb]{0.00,0.50,0.00}{##1}}}
\expandafter\def\csname PY@tok@kt\endcsname{\def\PY@tc##1{\textcolor[rgb]{0.69,0.00,0.25}{##1}}}
\expandafter\def\csname PY@tok@o\endcsname{\def\PY@tc##1{\textcolor[rgb]{0.40,0.40,0.40}{##1}}}
\expandafter\def\csname PY@tok@ow\endcsname{\let\PY@bf=\textbf\def\PY@tc##1{\textcolor[rgb]{0.67,0.13,1.00}{##1}}}
\expandafter\def\csname PY@tok@nb\endcsname{\def\PY@tc##1{\textcolor[rgb]{0.00,0.50,0.00}{##1}}}
\expandafter\def\csname PY@tok@nf\endcsname{\def\PY@tc##1{\textcolor[rgb]{0.00,0.00,1.00}{##1}}}
\expandafter\def\csname PY@tok@nc\endcsname{\let\PY@bf=\textbf\def\PY@tc##1{\textcolor[rgb]{0.00,0.00,1.00}{##1}}}
\expandafter\def\csname PY@tok@nn\endcsname{\let\PY@bf=\textbf\def\PY@tc##1{\textcolor[rgb]{0.00,0.00,1.00}{##1}}}
\expandafter\def\csname PY@tok@ne\endcsname{\let\PY@bf=\textbf\def\PY@tc##1{\textcolor[rgb]{0.82,0.25,0.23}{##1}}}
\expandafter\def\csname PY@tok@nv\endcsname{\def\PY@tc##1{\textcolor[rgb]{0.10,0.09,0.49}{##1}}}
\expandafter\def\csname PY@tok@no\endcsname{\def\PY@tc##1{\textcolor[rgb]{0.53,0.00,0.00}{##1}}}
\expandafter\def\csname PY@tok@nl\endcsname{\def\PY@tc##1{\textcolor[rgb]{0.63,0.63,0.00}{##1}}}
\expandafter\def\csname PY@tok@ni\endcsname{\let\PY@bf=\textbf\def\PY@tc##1{\textcolor[rgb]{0.60,0.60,0.60}{##1}}}
\expandafter\def\csname PY@tok@na\endcsname{\def\PY@tc##1{\textcolor[rgb]{0.49,0.56,0.16}{##1}}}
\expandafter\def\csname PY@tok@nt\endcsname{\let\PY@bf=\textbf\def\PY@tc##1{\textcolor[rgb]{0.00,0.50,0.00}{##1}}}
\expandafter\def\csname PY@tok@nd\endcsname{\def\PY@tc##1{\textcolor[rgb]{0.67,0.13,1.00}{##1}}}
\expandafter\def\csname PY@tok@s\endcsname{\def\PY@tc##1{\textcolor[rgb]{0.73,0.13,0.13}{##1}}}
\expandafter\def\csname PY@tok@sd\endcsname{\let\PY@it=\textit\def\PY@tc##1{\textcolor[rgb]{0.73,0.13,0.13}{##1}}}
\expandafter\def\csname PY@tok@si\endcsname{\let\PY@bf=\textbf\def\PY@tc##1{\textcolor[rgb]{0.73,0.40,0.53}{##1}}}
\expandafter\def\csname PY@tok@se\endcsname{\let\PY@bf=\textbf\def\PY@tc##1{\textcolor[rgb]{0.73,0.40,0.13}{##1}}}
\expandafter\def\csname PY@tok@sr\endcsname{\def\PY@tc##1{\textcolor[rgb]{0.73,0.40,0.53}{##1}}}
\expandafter\def\csname PY@tok@ss\endcsname{\def\PY@tc##1{\textcolor[rgb]{0.10,0.09,0.49}{##1}}}
\expandafter\def\csname PY@tok@sx\endcsname{\def\PY@tc##1{\textcolor[rgb]{0.00,0.50,0.00}{##1}}}
\expandafter\def\csname PY@tok@m\endcsname{\def\PY@tc##1{\textcolor[rgb]{0.40,0.40,0.40}{##1}}}
\expandafter\def\csname PY@tok@gh\endcsname{\let\PY@bf=\textbf\def\PY@tc##1{\textcolor[rgb]{0.00,0.00,0.50}{##1}}}
\expandafter\def\csname PY@tok@gu\endcsname{\let\PY@bf=\textbf\def\PY@tc##1{\textcolor[rgb]{0.50,0.00,0.50}{##1}}}
\expandafter\def\csname PY@tok@gd\endcsname{\def\PY@tc##1{\textcolor[rgb]{0.63,0.00,0.00}{##1}}}
\expandafter\def\csname PY@tok@gi\endcsname{\def\PY@tc##1{\textcolor[rgb]{0.00,0.63,0.00}{##1}}}
\expandafter\def\csname PY@tok@gr\endcsname{\def\PY@tc##1{\textcolor[rgb]{1.00,0.00,0.00}{##1}}}
\expandafter\def\csname PY@tok@ge\endcsname{\let\PY@it=\textit}
\expandafter\def\csname PY@tok@gs\endcsname{\let\PY@bf=\textbf}
\expandafter\def\csname PY@tok@gp\endcsname{\let\PY@bf=\textbf\def\PY@tc##1{\textcolor[rgb]{0.00,0.00,0.50}{##1}}}
\expandafter\def\csname PY@tok@go\endcsname{\def\PY@tc##1{\textcolor[rgb]{0.53,0.53,0.53}{##1}}}
\expandafter\def\csname PY@tok@gt\endcsname{\def\PY@tc##1{\textcolor[rgb]{0.00,0.27,0.87}{##1}}}
\expandafter\def\csname PY@tok@err\endcsname{\def\PY@bc##1{\setlength{\fboxsep}{0pt}\fcolorbox[rgb]{1.00,0.00,0.00}{1,1,1}{\strut ##1}}}
\expandafter\def\csname PY@tok@kc\endcsname{\let\PY@bf=\textbf\def\PY@tc##1{\textcolor[rgb]{0.00,0.50,0.00}{##1}}}
\expandafter\def\csname PY@tok@kd\endcsname{\let\PY@bf=\textbf\def\PY@tc##1{\textcolor[rgb]{0.00,0.50,0.00}{##1}}}
\expandafter\def\csname PY@tok@kn\endcsname{\let\PY@bf=\textbf\def\PY@tc##1{\textcolor[rgb]{0.00,0.50,0.00}{##1}}}
\expandafter\def\csname PY@tok@kr\endcsname{\let\PY@bf=\textbf\def\PY@tc##1{\textcolor[rgb]{0.00,0.50,0.00}{##1}}}
\expandafter\def\csname PY@tok@bp\endcsname{\def\PY@tc##1{\textcolor[rgb]{0.00,0.50,0.00}{##1}}}
\expandafter\def\csname PY@tok@fm\endcsname{\def\PY@tc##1{\textcolor[rgb]{0.00,0.00,1.00}{##1}}}
\expandafter\def\csname PY@tok@vc\endcsname{\def\PY@tc##1{\textcolor[rgb]{0.10,0.09,0.49}{##1}}}
\expandafter\def\csname PY@tok@vg\endcsname{\def\PY@tc##1{\textcolor[rgb]{0.10,0.09,0.49}{##1}}}
\expandafter\def\csname PY@tok@vi\endcsname{\def\PY@tc##1{\textcolor[rgb]{0.10,0.09,0.49}{##1}}}
\expandafter\def\csname PY@tok@vm\endcsname{\def\PY@tc##1{\textcolor[rgb]{0.10,0.09,0.49}{##1}}}
\expandafter\def\csname PY@tok@sa\endcsname{\def\PY@tc##1{\textcolor[rgb]{0.73,0.13,0.13}{##1}}}
\expandafter\def\csname PY@tok@sb\endcsname{\def\PY@tc##1{\textcolor[rgb]{0.73,0.13,0.13}{##1}}}
\expandafter\def\csname PY@tok@sc\endcsname{\def\PY@tc##1{\textcolor[rgb]{0.73,0.13,0.13}{##1}}}
\expandafter\def\csname PY@tok@dl\endcsname{\def\PY@tc##1{\textcolor[rgb]{0.73,0.13,0.13}{##1}}}
\expandafter\def\csname PY@tok@s2\endcsname{\def\PY@tc##1{\textcolor[rgb]{0.73,0.13,0.13}{##1}}}
\expandafter\def\csname PY@tok@sh\endcsname{\def\PY@tc##1{\textcolor[rgb]{0.73,0.13,0.13}{##1}}}
\expandafter\def\csname PY@tok@s1\endcsname{\def\PY@tc##1{\textcolor[rgb]{0.73,0.13,0.13}{##1}}}
\expandafter\def\csname PY@tok@mb\endcsname{\def\PY@tc##1{\textcolor[rgb]{0.40,0.40,0.40}{##1}}}
\expandafter\def\csname PY@tok@mf\endcsname{\def\PY@tc##1{\textcolor[rgb]{0.40,0.40,0.40}{##1}}}
\expandafter\def\csname PY@tok@mh\endcsname{\def\PY@tc##1{\textcolor[rgb]{0.40,0.40,0.40}{##1}}}
\expandafter\def\csname PY@tok@mi\endcsname{\def\PY@tc##1{\textcolor[rgb]{0.40,0.40,0.40}{##1}}}
\expandafter\def\csname PY@tok@il\endcsname{\def\PY@tc##1{\textcolor[rgb]{0.40,0.40,0.40}{##1}}}
\expandafter\def\csname PY@tok@mo\endcsname{\def\PY@tc##1{\textcolor[rgb]{0.40,0.40,0.40}{##1}}}
\expandafter\def\csname PY@tok@ch\endcsname{\let\PY@it=\textit\def\PY@tc##1{\textcolor[rgb]{0.25,0.50,0.50}{##1}}}
\expandafter\def\csname PY@tok@cm\endcsname{\let\PY@it=\textit\def\PY@tc##1{\textcolor[rgb]{0.25,0.50,0.50}{##1}}}
\expandafter\def\csname PY@tok@cpf\endcsname{\let\PY@it=\textit\def\PY@tc##1{\textcolor[rgb]{0.25,0.50,0.50}{##1}}}
\expandafter\def\csname PY@tok@c1\endcsname{\let\PY@it=\textit\def\PY@tc##1{\textcolor[rgb]{0.25,0.50,0.50}{##1}}}
\expandafter\def\csname PY@tok@cs\endcsname{\let\PY@it=\textit\def\PY@tc##1{\textcolor[rgb]{0.25,0.50,0.50}{##1}}}

\def\PYZbs{\char`\\}
\def\PYZus{\char`\_}
\def\PYZob{\char`\{}
\def\PYZcb{\char`\}}
\def\PYZca{\char`\^}
\def\PYZam{\char`\&}
\def\PYZlt{\char`\<}
\def\PYZgt{\char`\>}
\def\PYZsh{\char`\#}
\def\PYZpc{\char`\%}
\def\PYZdl{\char`\$}
\def\PYZhy{\char`\-}
\def\PYZsq{\char`\'}
\def\PYZdq{\char`\"}
\def\PYZti{\char`\~}
% for compatibility with earlier versions
\def\PYZat{@}
\def\PYZlb{[}
\def\PYZrb{]}
\makeatother


    % Exact colors from NB
    \definecolor{incolor}{rgb}{0.0, 0.0, 0.5}
    \definecolor{outcolor}{rgb}{0.545, 0.0, 0.0}



    
    % Prevent overflowing lines due to hard-to-break entities
    \sloppy 
    % Setup hyperref package
    \hypersetup{
      breaklinks=true,  % so long urls are correctly broken across lines
      colorlinks=true,
      urlcolor=urlcolor,
      linkcolor=linkcolor,
      citecolor=citecolor,
      }
    % Slightly bigger margins than the latex defaults
    
    \geometry{verbose,tmargin=1in,bmargin=1in,lmargin=1in,rmargin=1in}
    
    

    \begin{document}
    
    
    \maketitle
    
    

    
    \subsubsection{Rescheduled Class}\label{rescheduled-class}

You should all have received an email concerning the 2 rescheduled
class. Rescheduled classes will be held on: - 06/26 (June 26, Tuesday ),
period 1, room CALL 23 - 06/29 (June 29, Friday ), period 5, room CALL
23

You may attend either class.

These classes are for you to ask questions and get help with your
assignment.

There will be \emph{no new taught material} in these classes.

    \subsection{Please download the new class
notes.}\label{please-download-the-new-class-notes.}

\subsubsection{Step 1 : Navigate to the directory where your files are
stored.}\label{step-1-navigate-to-the-directory-where-your-files-are-stored.}

Open a terminal. Using \texttt{cd}, navigate to \emph{inside} the
ILAS\_Python\_for\_engineers folder on your computer. \#\#\# Step 3 :
Update the course notes by downloading the changes In the terminal type:

\begin{quote}
\texttt{git\ add\ -A\ git\ commit\ -m\ "commit"\ git\ fetch\ upstream\ git\ merge\ -X\ theirs\ upstream/master}
\end{quote}

    \section{Introduction to Simulation}\label{introduction-to-simulation}

 Recap: Solving Ordinary Differential Equations (ODE) Discontinuous
Functions    Example : A Mixing Tank Solving an ODE for a Specific
\emph{Solution} Value    Example : Reactor, Part 1 Solution to an ODE
evaluated at Specific \emph{Input} Values    Example : Reactor, Part 2
Matrix Multiplication Rules Useful Matrix Operations Systems of
Differential Equations Interactive Plots    Installing FFmpeg Animated
Plots    Example: Animating a Line    Example : Simulating Physical
Systems, Spring-Mass System    Example : Simulating Physical Systems,
Trajectory : Plotting the Object    Example : Simulating Physical
Systems, Trajectory : Plotting the Path    Example : Simulating Physical
Systems, Trajectory : Plotting the Path and Object Summary

    \subsubsection{Lesson Goal}\label{lesson-goal}

Produce simulations of systems to observe: - time-varying -
parameter-varying

effects.

\subsubsection{Fundamental programming
concepts}\label{fundamental-programming-concepts}

Employing computational image generation to represet real-life problems.

\begin{enumerate}
\def\labelenumi{\arabic{enumi}.}
\tightlist
\item
  Grpahical representation
\item
  Animated representation
\end{enumerate}

    \subsection{Recap: Solving Ordinary Differential Equations
(ODE)}\label{recap-solving-ordinary-differential-equations-ode}

 In the previous seminar we studied the function
\texttt{scipy.integrate.odeint}.

The function gives a numerical solution to a first order differential
equation for a given input parameter e.g. time.

Estimation methods can provide an easier way to solve problems than
finding an analytical soltion to the equation.

    \begin{Verbatim}[commandchars=\\\{\}]
{\color{incolor}In [{\color{incolor}2}]:} \PY{k+kn}{from} \PY{n+nn}{scipy}\PY{n+nn}{.}\PY{n+nn}{integrate} \PY{k}{import} \PY{n}{odeint}
        \PY{k+kn}{import} \PY{n+nn}{numpy} \PY{k}{as} \PY{n+nn}{np}
        \PY{k+kn}{import} \PY{n+nn}{matplotlib}\PY{n+nn}{.}\PY{n+nn}{pyplot} \PY{k}{as} \PY{n+nn}{plt}
\end{Verbatim}


    \textbf{Analytical solution (to an ODE) :} A function, which can be
solved to get a particular value.

e.g. solution to an ODE : \(y(x) = exp(x)\) We can \emph{solve} the
solution to find the value of \(x\) where \(y(x) = 2\).

Few practical systems lead to analytical solutions. Their use is limited
use.

\textbf{Numerical approximation (of an ODE) :} Independent variable and
approximation of corresponding function value at that variable.

    To recap last weeks class...

    The function \texttt{odeint} takes a user-defined function as an
argument.

This input function should define the derivative you want to solve in
the form:

\(f' = \frac{df}{dt} = g(t)\)

For example:

\(f' = cos(t), \quad f(t=0)=0\)

    \begin{Verbatim}[commandchars=\\\{\}]
{\color{incolor}In [{\color{incolor}3}]:} \PY{k}{def} \PY{n+nf}{dfdt}\PY{p}{(}\PY{n}{x}\PY{p}{,} \PY{n}{t}\PY{p}{)}\PY{p}{:}
            \PY{k}{return} \PY{n}{np}\PY{o}{.}\PY{n}{cos}\PY{p}{(}\PY{n}{t}\PY{p}{)}
\end{Verbatim}


    Two additional arguments are needed:

\begin{itemize}
\tightlist
\item
  the initial value of x
\item
  the value(s) of t at which to evaluate x, starting with the initial
  value
\item
  (if \texttt{dx\_dt} takes any other argumemts they are entered as a
  tuple as the third argument)
\end{itemize}

    \begin{Verbatim}[commandchars=\\\{\}]
{\color{incolor}In [{\color{incolor}4}]:} \PY{n}{ts} \PY{o}{=} \PY{n}{np}\PY{o}{.}\PY{n}{linspace}\PY{p}{(}\PY{l+m+mi}{0}\PY{p}{,}\PY{l+m+mi}{5}\PY{p}{,}\PY{l+m+mi}{100}\PY{p}{)} \PY{c+c1}{\PYZsh{} the value(s) of t at which to evaluate x}
        \PY{n}{f0} \PY{o}{=} \PY{l+m+mi}{0}                    \PY{c+c1}{\PYZsh{} the initial value}
        
        \PY{c+c1}{\PYZsh{} odeint returns x at each value of t}
        \PY{n}{fs} \PY{o}{=} \PY{n}{odeint}\PY{p}{(}\PY{n}{dfdt}\PY{p}{,} \PY{c+c1}{\PYZsh{} function}
                    \PY{n}{f0}\PY{p}{,}   \PY{c+c1}{\PYZsh{} initial value}
                    \PY{n}{ts}\PY{p}{)}   \PY{c+c1}{\PYZsh{} time span}
\end{Verbatim}


    The function \texttt{odeint} returns an \(n\times 1\) array (2D column).

To plot the output we have to "flatten" it to a 1D array.

Recall, \(f' = cos(t), \quad f = sin(t)\)

    \begin{Verbatim}[commandchars=\\\{\}]
{\color{incolor}In [{\color{incolor}5}]:} \PY{n}{fs} \PY{o}{=} \PY{n}{np}\PY{o}{.}\PY{n}{array}\PY{p}{(}\PY{n}{fs}\PY{p}{)}\PY{o}{.}\PY{n}{flatten}\PY{p}{(}\PY{p}{)}
\end{Verbatim}


    Plot the function

    \begin{Verbatim}[commandchars=\\\{\}]
{\color{incolor}In [{\color{incolor}6}]:} \PY{n}{plt}\PY{o}{.}\PY{n}{xlabel}\PY{p}{(}\PY{l+s+s2}{\PYZdq{}}\PY{l+s+s2}{t}\PY{l+s+s2}{\PYZdq{}}\PY{p}{)}
        \PY{n}{plt}\PY{o}{.}\PY{n}{ylabel}\PY{p}{(}\PY{l+s+s2}{\PYZdq{}}\PY{l+s+s2}{f}\PY{l+s+s2}{\PYZdq{}}\PY{p}{)}
        \PY{n}{plt}\PY{o}{.}\PY{n}{plot}\PY{p}{(}\PY{n}{ts}\PY{p}{,} \PY{n}{np}\PY{o}{.}\PY{n}{sin}\PY{p}{(}\PY{n}{ts}\PY{p}{)}\PY{p}{,} \PY{l+s+s1}{\PYZsq{}}\PY{l+s+s1}{c}\PY{l+s+s1}{\PYZsq{}}\PY{p}{,} \PY{n}{label}\PY{o}{=}\PY{l+s+s1}{\PYZsq{}}\PY{l+s+s1}{analytical solution}\PY{l+s+s1}{\PYZsq{}}\PY{p}{)}\PY{p}{;}
        \PY{n}{plt}\PY{o}{.}\PY{n}{plot}\PY{p}{(}\PY{n}{ts}\PY{p}{,} \PY{n}{fs}\PY{p}{,} \PY{l+s+s1}{\PYZsq{}}\PY{l+s+s1}{r\PYZhy{}\PYZhy{}}\PY{l+s+s1}{\PYZsq{}}\PY{p}{,} \PY{n}{label}\PY{o}{=}\PY{l+s+s1}{\PYZsq{}}\PY{l+s+s1}{numerical solution}\PY{l+s+s1}{\PYZsq{}}\PY{p}{)}\PY{p}{;}
        \PY{n}{plt}\PY{o}{.}\PY{n}{legend}\PY{p}{(}\PY{n}{loc}\PY{o}{=}\PY{l+s+s1}{\PYZsq{}}\PY{l+s+s1}{best}\PY{l+s+s1}{\PYZsq{}}\PY{p}{)}
\end{Verbatim}


\begin{Verbatim}[commandchars=\\\{\}]
{\color{outcolor}Out[{\color{outcolor}6}]:} <matplotlib.legend.Legend at 0x114b6ce48>
\end{Verbatim}
            
    \begin{center}
    \adjustimage{max size={0.9\linewidth}{0.9\paperheight}}{output_15_1.png}
    \end{center}
    { \hspace*{\fill} \\}
    
    The numerical and analytical solutions agree.

    \subsection{Discontinuous Functions}\label{discontinuous-functions}

 The computational solver to loops through the range of input vlaues.

It estimates the value of the function at each input value.

This makes it easy to apply the solver to piecewise or discontinous
functions.

    An example of where this is useful is where a \emph{forcing function}
effects your solution.

    \subparagraph{Example : A Mixing Tank}\label{example-a-mixing-tank}

A tank initially contains a mixture: - 300 g of salt - 1000 L of water.

From t=0 min: - inlet solution 4 g/L (salt/water) - inlet flow rate 6
L/min.

From t=10 min: - inlet solution 2 g/L (salt/water) - inlet flow rate 6
L/min (unchanged)

Oulet flow rate = inlet floow rate = 6 L/min (unchanged).

The contents are mixed thoroughly.

    A differential equation to represent the change in the mass of salt in
the tank with respect to time:

\begin{align*}
\frac{dM_S}{dt} &= u_{in}C_{in}(t) - u_{out}C_{out}(t)\\
                &= u_{in}C_{in}(t)-u_{out}\frac{M_S}{V} \\
\end{align*}

where \(u\) = inlet flow rate \(V\) = volume of liquid in tank
\(C_{in}\) = salt concentration inflow g/L \(M_S\) = mass of salt in
tank

Initial condition: \(M_S(t=0) = 300\)

    We can first write a function to represent the change of \(C_{in}\) with
time:

    \begin{Verbatim}[commandchars=\\\{\}]
{\color{incolor}In [{\color{incolor}7}]:} \PY{k}{def} \PY{n+nf}{Cs}\PY{p}{(}\PY{n}{t}\PY{p}{)}\PY{p}{:}
            \PY{l+s+s1}{\PYZsq{}}\PY{l+s+s1}{inlet concentration of salt}\PY{l+s+s1}{\PYZsq{}}
            \PY{k}{if} \PY{n}{t} \PY{o}{\PYZlt{}} \PY{l+m+mi}{0}\PY{p}{:}
                \PY{n}{cs} \PY{o}{=} \PY{l+m+mf}{0.0} \PY{c+c1}{\PYZsh{} g/L}
            \PY{k}{elif} \PY{p}{(}\PY{n}{t} \PY{o}{\PYZgt{}} \PY{l+m+mi}{0}\PY{p}{)} \PY{o+ow}{and} \PY{p}{(}\PY{n}{t} \PY{o}{\PYZlt{}}\PY{o}{=} \PY{l+m+mi}{10}\PY{p}{)}\PY{p}{:}
                \PY{n}{cs} \PY{o}{=} \PY{l+m+mf}{4.0} 
            \PY{k}{else}\PY{p}{:}
                \PY{n}{cs} \PY{o}{=} \PY{l+m+mf}{2.0}
            \PY{k}{return} \PY{n}{cs}
\end{Verbatim}


    Then we write a function to define \(\frac{dM_S}{dt}\) just as we did in
the last example.

    \begin{Verbatim}[commandchars=\\\{\}]
{\color{incolor}In [{\color{incolor}8}]:} \PY{n}{V} \PY{o}{=} \PY{l+m+mf}{1000.0} \PY{c+c1}{\PYZsh{} volume of liquid in tank}
        \PY{n}{u} \PY{o}{=} \PY{l+m+mf}{6.0}    \PY{c+c1}{\PYZsh{} inflow and outflow rate}
        
        \PY{k}{def} \PY{n+nf}{salt\PYZus{}mass}\PY{p}{(}\PY{n}{Ms}\PY{p}{,} \PY{n}{t}\PY{p}{)}\PY{p}{:}
            \PY{l+s+s1}{\PYZsq{}}\PY{l+s+s1}{mass of salt in the tank}\PY{l+s+s1}{\PYZsq{}}
            \PY{n}{dMsdt} \PY{o}{=} \PY{n}{u} \PY{o}{*} \PY{n}{Cs}\PY{p}{(}\PY{n}{t}\PY{p}{)} \PY{o}{\PYZhy{}} \PY{n}{u} \PY{o}{*} \PY{n}{Ms} \PY{o}{/} \PY{n}{V}
            \PY{k}{return} \PY{n}{dMsdt}
\end{Verbatim}


    \begin{Verbatim}[commandchars=\\\{\}]
{\color{incolor}In [{\color{incolor}9}]:} \PY{c+c1}{\PYZsh{} initial value (at t=0)}
        \PY{n}{M0} \PY{o}{=} \PY{l+m+mi}{300} \PY{c+c1}{\PYZsh{} g salt}
        
        \PY{c+c1}{\PYZsh{} time span }
        \PY{n}{tspan} \PY{o}{=} \PY{n}{np}\PY{o}{.}\PY{n}{linspace}\PY{p}{(}\PY{l+m+mf}{0.0}\PY{p}{,} \PY{l+m+mf}{15.0}\PY{p}{,} \PY{l+m+mi}{50}\PY{p}{)}
        
        \PY{c+c1}{\PYZsh{} mass of salt in tank over time span}
        \PY{n}{Ms} \PY{o}{=} \PY{n}{odeint}\PY{p}{(}\PY{n}{salt\PYZus{}mass}\PY{p}{,} \PY{n}{M0}\PY{p}{,} \PY{n}{tspan}\PY{p}{)}
\end{Verbatim}


    Plot the solution to observe the change in the rate of increase in salt
concentration at \(t=10\)s.

    \begin{Verbatim}[commandchars=\\\{\}]
{\color{incolor}In [{\color{incolor}10}]:} \PY{n}{plt}\PY{o}{.}\PY{n}{plot}\PY{p}{(}\PY{n}{tspan}\PY{p}{,} \PY{n}{Ms}\PY{o}{/}\PY{n}{V}\PY{p}{,} \PY{l+s+s1}{\PYZsq{}}\PY{l+s+s1}{b.}\PY{l+s+s1}{\PYZsq{}}\PY{p}{)}
         \PY{n}{plt}\PY{o}{.}\PY{n}{xlabel}\PY{p}{(}\PY{l+s+s1}{\PYZsq{}}\PY{l+s+s1}{Time (min)}\PY{l+s+s1}{\PYZsq{}}\PY{p}{)}
         \PY{n}{plt}\PY{o}{.}\PY{n}{ylabel}\PY{p}{(}\PY{l+s+s1}{\PYZsq{}}\PY{l+s+s1}{Salt concentration (g/L)}\PY{l+s+s1}{\PYZsq{}}\PY{p}{)}
\end{Verbatim}


\begin{Verbatim}[commandchars=\\\{\}]
{\color{outcolor}Out[{\color{outcolor}10}]:} Text(0,0.5,'Salt concentration (g/L)')
\end{Verbatim}
            
    \begin{center}
    \adjustimage{max size={0.9\linewidth}{0.9\paperheight}}{output_27_1.png}
    \end{center}
    { \hspace*{\fill} \\}
    
    \subsection{\texorpdfstring{Solving an ODE for a Specific
\emph{Solution}
Value.}{Solving an ODE for a Specific Solution Value.}}\label{solving-an-ode-for-a-specific-solution-value.}

 The analytical solution to an ODE is a function the value of input
variable \(x\) that makes \(y(x)\) equal to some value of interest.

    In a numerical solution to an ODE we get a vector of independent
variable values, and the corresponding function values at those values.

This may not include the function value of itnerest.

To solve for a particular function value we need a dfferent approach.

    One approach to this is using interpolation, which we studied in
\texttt{05\_Plotting\_Analysing.ipynb}...

    \subsubsection{Example : Reactor, Part 1}\label{example-reactor-part-1}

 The rate of change of the concentration \(C_A\) of a substance within a
reactor is:

\[\frac{dC_A}{dt}=-kC_A^2 \]

where \(k = 0.23 \quad \textrm{L/mol/s}\)

Initial condition: \(C_A(t=0) = 2.3 \quad \textrm{mol/L}\)

Compute the time it takes for \(C_A\) to be reduced to
\(1 \textrm{mol/L}\).

\emph{i.e. Find input variable \(t\) when output variable \(C_A = 1\)}

    \begin{Verbatim}[commandchars=\\\{\}]
{\color{incolor}In [{\color{incolor}11}]:} \PY{n}{k} \PY{o}{=} \PY{l+m+mf}{0.23}
         \PY{n}{Ca0} \PY{o}{=} \PY{l+m+mf}{2.3}  \PY{c+c1}{\PYZsh{} initial condition}
         
         \PY{k}{def} \PY{n+nf}{dCadt}\PY{p}{(}\PY{n}{Ca}\PY{p}{,} \PY{n}{t}\PY{p}{)}\PY{p}{:}
             \PY{k}{return} \PY{o}{\PYZhy{}}\PY{n}{k} \PY{o}{*} \PY{n}{Ca}\PY{o}{*}\PY{o}{*}\PY{l+m+mi}{2}
         
         \PY{n}{tspan} \PY{o}{=} \PY{n}{np}\PY{o}{.}\PY{n}{linspace}\PY{p}{(}\PY{l+m+mi}{0}\PY{p}{,} \PY{l+m+mi}{10}\PY{p}{,} \PY{l+m+mi}{14}\PY{p}{)}  \PY{c+c1}{\PYZsh{} range of time values}
         \PY{n}{sol} \PY{o}{=} \PY{n}{odeint}\PY{p}{(}\PY{n}{dCadt}\PY{p}{,} \PY{n}{Ca0}\PY{p}{,} \PY{n}{tspan}\PY{p}{)}
\end{Verbatim}


    \begin{Verbatim}[commandchars=\\\{\}]
{\color{incolor}In [{\color{incolor}12}]:} \PY{n}{plt}\PY{o}{.}\PY{n}{plot}\PY{p}{(}\PY{n}{tspan}\PY{p}{,} \PY{n}{sol}\PY{p}{)}
         \PY{n}{plt}\PY{o}{.}\PY{n}{xlabel}\PY{p}{(}\PY{l+s+s1}{\PYZsq{}}\PY{l+s+s1}{Time (s)}\PY{l+s+s1}{\PYZsq{}}\PY{p}{)}
         \PY{n}{plt}\PY{o}{.}\PY{n}{ylabel}\PY{p}{(}\PY{l+s+s1}{\PYZsq{}}\PY{l+s+s1}{\PYZdl{}C\PYZus{}A\PYZdl{} (mol/L)}\PY{l+s+s1}{\PYZsq{}}\PY{p}{)}
\end{Verbatim}


\begin{Verbatim}[commandchars=\\\{\}]
{\color{outcolor}Out[{\color{outcolor}12}]:} Text(0,0.5,'\$C\_A\$ (mol/L)')
\end{Verbatim}
            
    \begin{center}
    \adjustimage{max size={0.9\linewidth}{0.9\paperheight}}{output_33_1.png}
    \end{center}
    { \hspace*{\fill} \\}
    
    We can see that the solution (\(C_A = 1\)) is close to 2s.

    We create an interpolating function to estimate the solution.

First, observe how the solution is represented.

    \begin{Verbatim}[commandchars=\\\{\}]
{\color{incolor}In [{\color{incolor}13}]:} \PY{n+nb}{print}\PY{p}{(}\PY{n}{sol}\PY{p}{,} \PY{n+nb}{type}\PY{p}{(}\PY{n}{sol}\PY{p}{)}\PY{p}{)}
\end{Verbatim}


    \begin{Verbatim}[commandchars=\\\{\}]
[[2.3       ]
 [1.63477314]
 [1.26802375]
 [1.03567716]
 [0.87529271]
 [0.75792139]
 [0.66830573]
 [0.59764138]
 [0.54049165]
 [0.4933179 ]
 [0.45371772]
 [0.42000278]
 [0.39095185]
 [0.36565974]] <class 'numpy.ndarray'>

    \end{Verbatim}

    We can plot this range without issue.

However, when interpolating, we must specify the index of the range of
values:

    \begin{Verbatim}[commandchars=\\\{\}]
{\color{incolor}In [{\color{incolor}15}]:} \PY{n}{Ca} \PY{o}{=} \PY{n}{sol}\PY{p}{[}\PY{p}{:}\PY{p}{,}\PY{l+m+mi}{0}\PY{p}{]}
\end{Verbatim}


    \begin{Verbatim}[commandchars=\\\{\}]
{\color{incolor}In [{\color{incolor}16}]:} \PY{c+c1}{\PYZsh{} interpolate : Method 1}
         \PY{k+kn}{from} \PY{n+nn}{scipy}\PY{n+nn}{.}\PY{n+nn}{interpolate} \PY{k}{import} \PY{n}{interp1d}
         
         \PY{n}{ca\PYZus{}func} \PY{o}{=} \PY{n}{interp1d}\PY{p}{(}\PY{n}{tspan}\PY{p}{,} \PY{n}{Ca}\PY{p}{,} \PY{l+s+s1}{\PYZsq{}}\PY{l+s+s1}{cubic}\PY{l+s+s1}{\PYZsq{}}\PY{p}{)}
         
         
         \PY{c+c1}{\PYZsh{} \PYZsh{} interpolate : Method 2}
         \PY{c+c1}{\PYZsh{} from scipy.interpolate import splrep}
         \PY{c+c1}{\PYZsh{} from scipy.interpolate import splev}
         
         \PY{c+c1}{\PYZsh{} order\PYZus{}poly = 3}
         \PY{c+c1}{\PYZsh{} ca\PYZus{}func = splrep(tspan, Ca, k=order\PYZus{}poly)}
\end{Verbatim}


    \begin{Verbatim}[commandchars=\\\{\}]
{\color{incolor}In [{\color{incolor}241}]:} \PY{c+c1}{\PYZsh{} plot the original data }
          \PY{n}{plt}\PY{o}{.}\PY{n}{plot}\PY{p}{(}\PY{n}{tspan}\PY{p}{,} \PY{n}{Ca}\PY{p}{,} \PY{n}{label}\PY{o}{=}\PY{l+s+s1}{\PYZsq{}}\PY{l+s+s1}{original data}\PY{l+s+s1}{\PYZsq{}}\PY{p}{)}
          
          \PY{c+c1}{\PYZsh{} plot the interpolated data}
          \PY{n}{tspan\PYZus{}i} \PY{o}{=} \PY{n}{np}\PY{o}{.}\PY{n}{linspace}\PY{p}{(}\PY{l+m+mi}{0}\PY{p}{,} \PY{l+m+mi}{10}\PY{p}{,} \PY{l+m+mi}{200}\PY{p}{)}   \PY{c+c1}{\PYZsh{} create a finer range of time variables}
          
          \PY{c+c1}{\PYZsh{} Method 1}
          \PY{n}{plt}\PY{o}{.}\PY{n}{plot}\PY{p}{(}\PY{n}{tspan\PYZus{}i}\PY{p}{,} \PY{n}{ca\PYZus{}func}\PY{p}{(}\PY{n}{tspan\PYZus{}i}\PY{p}{)}\PY{p}{,} \PY{l+s+s1}{\PYZsq{}}\PY{l+s+s1}{r\PYZhy{}\PYZhy{}}\PY{l+s+s1}{\PYZsq{}}\PY{p}{,} \PY{n}{label}\PY{o}{=}\PY{l+s+s1}{\PYZsq{}}\PY{l+s+s1}{interpolation}\PY{l+s+s1}{\PYZsq{}}\PY{p}{)} \PY{c+c1}{\PYZsh{} Method 1}
          
          \PY{c+c1}{\PYZsh{} Method 2}
          \PY{c+c1}{\PYZsh{}plt.plot(tspan\PYZus{}i, splev(tspan\PYZus{}i, ca\PYZus{}func), \PYZsq{}r\PYZhy{}\PYZhy{}\PYZsq{}, label=\PYZsq{}interpolation\PYZsq{}) \PYZsh{} Method 2}
          
          \PY{n}{plt}\PY{o}{.}\PY{n}{xlabel}\PY{p}{(}\PY{l+s+s1}{\PYZsq{}}\PY{l+s+s1}{Time (s)}\PY{l+s+s1}{\PYZsq{}}\PY{p}{)}
          \PY{n}{plt}\PY{o}{.}\PY{n}{ylabel}\PY{p}{(}\PY{l+s+s1}{\PYZsq{}}\PY{l+s+s1}{\PYZdl{}C\PYZus{}A\PYZdl{} (mol/L)}\PY{l+s+s1}{\PYZsq{}}\PY{p}{)}
          \PY{n}{plt}\PY{o}{.}\PY{n}{legend}\PY{p}{(}\PY{p}{)}
\end{Verbatim}


\begin{Verbatim}[commandchars=\\\{\}]
{\color{outcolor}Out[{\color{outcolor}241}]:} <matplotlib.legend.Legend at 0x181ed230b8>
\end{Verbatim}
            
    \begin{center}
    \adjustimage{max size={0.9\linewidth}{0.9\paperheight}}{output_40_1.png}
    \end{center}
    { \hspace*{\fill} \\}
    
    The interpolated data provides a good fit.

Now we solve the problem : find time, \(t\) when \(C_A = 1\)

Remember, our approximation : \(t \approx 2\) when \(C_A = 1\)

    \begin{Verbatim}[commandchars=\\\{\}]
{\color{incolor}In [{\color{incolor}17}]:} \PY{n}{tguess} \PY{o}{=} \PY{l+m+mf}{2.0}
         
         \PY{k}{def} \PY{n+nf}{func}\PY{p}{(}\PY{n}{t}\PY{p}{)}\PY{p}{:}
             \PY{l+s+s1}{\PYZsq{}}\PY{l+s+s1}{returns 0 when interpolated function ca\PYZus{}func(t) = desired value, 1}\PY{l+s+s1}{\PYZsq{}}
             \PY{k}{return} \PY{l+m+mf}{1.0} \PY{o}{\PYZhy{}} \PY{n}{ca\PYZus{}func}\PY{p}{(}\PY{n}{t}\PY{p}{)}
\end{Verbatim}


    In \texttt{07\_RootFinding\_CoupledEquations.ipynb} we learnt to use the
function \texttt{fsolve} for estimating the roots of a function given an
initial estimate.

Then we use \texttt{fsolve} to solve the function (find where function
returns 0), usign initial guess = 2.

    \begin{Verbatim}[commandchars=\\\{\}]
{\color{incolor}In [{\color{incolor}243}]:} \PY{k+kn}{from} \PY{n+nn}{scipy}\PY{n+nn}{.}\PY{n+nn}{optimize} \PY{k}{import} \PY{n}{fsolve}
          
          \PY{n}{tsol2} \PY{o}{=} \PY{n}{fsolve}\PY{p}{(}\PY{n}{func}\PY{p}{,} \PY{n}{tguess}\PY{p}{)}
          \PY{c+c1}{\PYZsh{} tsol2, = fsolve(func, tguess)  \PYZsh{} using a comma extracts a single varibale stored as array}
          \PY{n+nb}{print}\PY{p}{(}\PY{n}{tsol2}\PY{p}{)}
\end{Verbatim}


    \begin{Verbatim}[commandchars=\\\{\}]
[2.45688986]

    \end{Verbatim}

    We can express this more concisely as a \texttt{lambda} function:

    \begin{Verbatim}[commandchars=\\\{\}]
{\color{incolor}In [{\color{incolor}244}]:} \PY{c+c1}{\PYZsh{} input t}
          \PY{c+c1}{\PYZsh{} output 1.0 \PYZhy{} ca\PYZus{}func(t}
          \PY{n}{tsol}\PY{p}{,} \PY{o}{=} \PY{n}{fsolve}\PY{p}{(}\PY{k}{lambda} \PY{n}{t}\PY{p}{:} \PY{l+m+mf}{1.0} \PY{o}{\PYZhy{}} \PY{n}{ca\PYZus{}func}\PY{p}{(}\PY{n}{t}\PY{p}{)}\PY{p}{,} \PY{n}{tguess}\PY{p}{)}
          \PY{n+nb}{print}\PY{p}{(}\PY{n}{tsol}\PY{p}{)}
\end{Verbatim}


    \begin{Verbatim}[commandchars=\\\{\}]
2.4568898598165143

    \end{Verbatim}

    \subsection{\texorpdfstring{Solution to an ODE evaulated at Specific
\emph{Input}
Values.}{Solution to an ODE evaulated at Specific Input Values.}}\label{solution-to-an-ode-evaulated-at-specific-input-values.}

 Sometimes it is desirable to get the solution at specific points e.g.
at \(t = [0, 0.2, 0.4, 0.8]\)

For example we might want to compare experimental measurements at those
time points.

    To find the solution for specific input values, we simply set the range
of input variables to the values of interest.

    \subsubsection{Example : Reactor, Part 2}\label{example-reactor-part-2}

 The rate of change of the concentration \(C_A\) of a substance within a
reactor is:

\[\frac{dC_A}{dt}=-kC_A^2 \]

where \(k = 0.23 \quad \textrm{L/mol/s}\)

Initial condition: \(C_A(t=0) = 2.3 \quad \textrm{mol/L}\)

Compute the contration \(C_A\) at time \(t = [0, 2, 4, 8]\).

    We can use the same function as in \textbf{Example : Reactor, Part 1}

    \begin{Verbatim}[commandchars=\\\{\}]
{\color{incolor}In [{\color{incolor}245}]:} \PY{n}{k} \PY{o}{=} \PY{l+m+mf}{0.23}
          \PY{n}{Ca0} \PY{o}{=} \PY{l+m+mf}{2.3}
          
          \PY{k}{def} \PY{n+nf}{dCadt}\PY{p}{(}\PY{n}{Ca}\PY{p}{,} \PY{n}{t}\PY{p}{)}\PY{p}{:}
              \PY{k}{return} \PY{o}{\PYZhy{}}\PY{n}{k} \PY{o}{*} \PY{n}{Ca}\PY{o}{*}\PY{o}{*}\PY{l+m+mi}{2}
\end{Verbatim}


    ...we just apply the function to the new range of time values:

    \begin{Verbatim}[commandchars=\\\{\}]
{\color{incolor}In [{\color{incolor}246}]:} \PY{c+c1}{\PYZsh{} previous problem solution}
          \PY{n}{tspan} \PY{o}{=} \PY{n}{np}\PY{o}{.}\PY{n}{linspace}\PY{p}{(}\PY{l+m+mi}{0}\PY{p}{,} \PY{l+m+mi}{10}\PY{p}{,} \PY{l+m+mi}{14}\PY{p}{)}
          \PY{n}{plt}\PY{o}{.}\PY{n}{plot}\PY{p}{(}\PY{n}{tspan}\PY{p}{,} \PY{n}{odeint}\PY{p}{(}\PY{n}{dCadt}\PY{p}{,} \PY{n}{Ca0}\PY{p}{,} \PY{n}{tspan}\PY{p}{)}\PY{p}{,} \PY{n}{label}\PY{o}{=}\PY{l+s+s1}{\PYZsq{}}\PY{l+s+s1}{range time}\PY{l+s+s1}{\PYZsq{}}\PY{p}{)}
          
          \PY{c+c1}{\PYZsh{} specific value problem}
          \PY{n}{tspan} \PY{o}{=} \PY{p}{[}\PY{l+m+mi}{0}\PY{p}{,} \PY{l+m+mi}{2}\PY{p}{,} \PY{l+m+mi}{4}\PY{p}{,} \PY{l+m+mi}{8}\PY{p}{]}
          \PY{n}{plt}\PY{o}{.}\PY{n}{plot}\PY{p}{(}\PY{n}{tspan}\PY{p}{,} \PY{n}{odeint}\PY{p}{(}\PY{n}{dCadt}\PY{p}{,} \PY{n}{Ca0}\PY{p}{,} \PY{n}{tspan}\PY{p}{)}\PY{p}{,} \PY{l+s+s1}{\PYZsq{}}\PY{l+s+s1}{ro}\PY{l+s+s1}{\PYZsq{}}\PY{p}{,} \PY{n}{label}\PY{o}{=}\PY{l+s+s1}{\PYZsq{}}\PY{l+s+s1}{range time}\PY{l+s+s1}{\PYZsq{}}\PY{p}{)}
          
          \PY{n}{plt}\PY{o}{.}\PY{n}{xlabel}\PY{p}{(}\PY{l+s+s1}{\PYZsq{}}\PY{l+s+s1}{Time (s)}\PY{l+s+s1}{\PYZsq{}}\PY{p}{)}
          \PY{n}{plt}\PY{o}{.}\PY{n}{ylabel}\PY{p}{(}\PY{l+s+s1}{\PYZsq{}}\PY{l+s+s1}{\PYZdl{}C\PYZus{}A\PYZdl{} (mol/L)}\PY{l+s+s1}{\PYZsq{}}\PY{p}{)}
\end{Verbatim}


\begin{Verbatim}[commandchars=\\\{\}]
{\color{outcolor}Out[{\color{outcolor}246}]:} Text(0,0.5,'\$C\_A\$ (mol/L)')
\end{Verbatim}
            
    \begin{center}
    \adjustimage{max size={0.9\linewidth}{0.9\paperheight}}{output_53_1.png}
    \end{center}
    { \hspace*{\fill} \\}
    
    \begin{Verbatim}[commandchars=\\\{\}]
{\color{incolor}In [{\color{incolor}247}]:} \PY{n+nb}{print}\PY{p}{(}\PY{n}{odeint}\PY{p}{(}\PY{n}{dCadt}\PY{p}{,} \PY{n}{Ca0}\PY{p}{,} \PY{n}{tspan}\PY{p}{)}\PY{p}{)}
          
          \PY{c+c1}{\PYZsh{} use index to access column 0}
          \PY{c+c1}{\PYZsh{} print(odeint(dCadt, Ca0, tspan)[:,0])}
\end{Verbatim}


    \begin{Verbatim}[commandchars=\\\{\}]
[[2.3       ]
 [1.11758989]
 [0.73812577]
 [0.43960241]]

    \end{Verbatim}

    \subsection{Matrix Multiplication
Rules.}\label{matrix-multiplication-rules.}

 The following section explains the process of matrix multiplication in
case this is an unfamiliar topic for you.

Read this section to understand the use of matrices in the examples that
follow.

    If the number of \textbf{columns in A} is the same as number of
\textbf{rows in B}, we can find the matrix product of \(\mathbf{A}\) and
\(\mathbf{B}\). \(\mathbf{C} = \mathbf{A} \cdot \mathbf{B}\)

    For example: \(\mathbf{A}\) has 3 rows and \textbf{3 columns}
\(\mathbf{B}\) has \textbf{3 rows} and 1 column (\(\mathbf{B}\) is a
vector represented as a matrix)

\begin{equation*}
\underbrace{
\begin{bmatrix}
1 & 2 & 3 \\
4 & 5 & 6 \\
7 & 8 & 9 \\
\end{bmatrix}
}_{\mathbf{A} \text{ 3 rows} \text{ 3 columns}}
\cdot
\underbrace{
\begin{bmatrix}
10 \\
20 \\
30 \\
\end{bmatrix}
}_{\mathbf{B} \text{  3 rows} \text{  1 column}}
\end{equation*}

So we can multiply them...

    In matrix \(\mathbf{C}\), the element in \textbf{row \(i\)},
\textbf{column \(j\)}

is equal to the dot product of the \(i\)th \textbf{row} of
\(\mathbf{A}\), \(j\)th \textbf{column} of \(\mathbf{B}\).m

    \begin{equation*}
\underbrace{
\begin{bmatrix}
\color{red}1 & \color{red}2 & \color{red}3 \\
4 & 5 & 6 \\
7 & 8 & 9 \\
\end{bmatrix}
}_{\mathbf{A} \text{ 3 rows} \text{ 3 columns}}
\cdot
\underbrace{
\begin{bmatrix}
\color{red}{10} \\
\color{red}{20} \\
\color{red}{30} \\
\end{bmatrix}
}_{\mathbf{B} \text{  3 rows} \text{  1 column}}
=\underbrace{
\begin{bmatrix}
\color{red}{1 \cdot 10 \quad + \quad 2 \cdot 20 \quad + \quad 3 \cdot 30} \\
4 \cdot 10 \quad + \quad 5 \cdot 20 \quad + \quad 6 \cdot 30 \\
7 \cdot 10 \quad + \quad 8 \cdot 20 \quad + \quad 9 \cdot 30 \\
\end{bmatrix}
}_{\mathbf{C} \text{  3 rows} \text{  1 column}}
=\underbrace{
\begin{bmatrix}
\color{red}{140} \\
320 \\
500 \\
\end{bmatrix}
}_{\mathbf{C} \text{  3 rows} \text{  1 column1}}
\end{equation*}

    \begin{equation*}
\underbrace{
\begin{bmatrix}
1 & 2 & 3 \\
\color{red}4 & \color{red}5 & \color{red}6 \\
7 & 8 & 9 \\
\end{bmatrix}
}_{\mathbf{A} \text{ 3 rows} \text{ 3 columns}}
\cdot
\underbrace{
\begin{bmatrix}
\color{red}{10} \\
\color{red}{20} \\
\color{red}{30} \\
\end{bmatrix}
}_{\mathbf{B} \text{  3 rows} \text{  1 column}}
=\underbrace{
\begin{bmatrix}
1 \cdot 10 \quad + \quad 2 \cdot 20 \quad + \quad 3 \cdot 30 \\
\color{red}{4 \cdot 10 \quad + \quad 5 \cdot 20 \quad + \quad 6 \cdot 30} \\
7 \cdot 10 \quad + \quad 8 \cdot 20 \quad + \quad 9 \cdot 30 \\
\end{bmatrix}
}_{\mathbf{C} \text{  3 rows} \text{  1 column}}
=\underbrace{
\begin{bmatrix}
140 \\
\color{red}{320} \\
500 \\
\end{bmatrix}
}_{\mathbf{C} \text{  3 rows} \text{  1 column1}}
\end{equation*}

    \begin{equation*}
\underbrace{
\begin{bmatrix}
1 & 2 & 3 \\
4 & 5 & 6 \\
\color{red}7 & \color{red}8 & \color{red}9 \\
\end{bmatrix}
}_{\mathbf{A} \text{ 3 rows} \text{ 3 columns}}
\cdot
\underbrace{
\begin{bmatrix}
\color{red}{10} \\
\color{red}{20} \\
\color{red}{30} \\
\end{bmatrix}
}_{\mathbf{B} \text{  3 rows} \text{  1 column}}
=\underbrace{
\begin{bmatrix}
1 \cdot 10 \quad + \quad 2 \cdot 20 \quad + \quad 3 \cdot 30 \\
4 \cdot 10 \quad + \quad 5 \cdot 20 \quad + \quad 6 \cdot 30 \\
\color{red}{7 \cdot 10 \quad + \quad 8 \cdot 20 \quad + \quad 9 \cdot 30} \\
\end{bmatrix}
}_{\mathbf{C} \text{  3 rows} \text{  1 column}}
=\underbrace{
\begin{bmatrix}
140 \\
320 \\
\color{red}{500} \\
\end{bmatrix}
}_{\mathbf{C} \text{  3 rows} \text{  1 column1}}
\end{equation*}

    \begin{equation*}
\underbrace{
\begin{bmatrix}
1 & 2 & 3 \\
4 & 5 & 6 \\
7 & 8 & 9 \\
\end{bmatrix}
}_{\mathbf{A} \text{ 3 rows} \text{ 3 columns}}
\cdot
\underbrace{
\begin{bmatrix}
10 \\
20 \\
30 \\
\end{bmatrix}
}_{\mathbf{B} \text{  3 rows} \text{  1 column}}
=\underbrace{
\begin{bmatrix}
1 \cdot 10 \quad + \quad 2 \cdot 20 \quad + \quad 3 \cdot 30 \\
4 \cdot 10 \quad + \quad 5 \cdot 20 \quad + \quad 6 \cdot 30 \\
7 \cdot 10 \quad + \quad 8 \cdot 20 \quad + \quad 9 \cdot 30 \\
\end{bmatrix}
}_{\mathbf{C} \text{  3 rows} \text{  1 column}}
=\underbrace{
\begin{bmatrix}
140 \\
320 \\
500 \\
\end{bmatrix}
}_{\mathbf{C} \text{  3 rows} \text{  1 column1}}
\end{equation*}

Matrix \(\mathbf{C}\) therefore has: - the same number of \textbf{rows}
as \(\mathbf{A}\), - the same number of \textbf{columns} as
\(\mathbf{B}\).

    \begin{Verbatim}[commandchars=\\\{\}]
{\color{incolor}In [{\color{incolor}248}]:} \PY{c+c1}{\PYZsh{} In the equation above, vector B must be represented as a column vector}
          \PY{n}{A} \PY{o}{=} \PY{n}{np}\PY{o}{.}\PY{n}{array}\PY{p}{(}\PY{p}{[}\PY{p}{[}\PY{l+m+mi}{1}\PY{p}{,} \PY{l+m+mi}{2}\PY{p}{,} \PY{l+m+mi}{3}\PY{p}{]}\PY{p}{,} 
                        \PY{p}{[}\PY{l+m+mi}{4}\PY{p}{,} \PY{l+m+mi}{5}\PY{p}{,} \PY{l+m+mi}{6}\PY{p}{]}\PY{p}{,}
                        \PY{p}{[}\PY{l+m+mi}{7}\PY{p}{,} \PY{l+m+mi}{8}\PY{p}{,} \PY{l+m+mi}{9}\PY{p}{]}\PY{p}{]}\PY{p}{)}
          
          
          \PY{c+c1}{\PYZsh{} In Python, 1D arrays are ALWAYS represented horizontally }
          \PY{c+c1}{\PYZsh{} This does not define the array as a row vector}
          \PY{n}{B} \PY{o}{=} \PY{n}{np}\PY{o}{.}\PY{n}{array}\PY{p}{(}\PY{p}{[}\PY{l+m+mi}{10}\PY{p}{,} \PY{l+m+mi}{20}\PY{p}{,} \PY{l+m+mi}{30}\PY{p}{]}\PY{p}{)}
          
          
          \PY{c+c1}{\PYZsh{} For example, C is represented horizontally }
          \PY{n}{C} \PY{o}{=} \PY{n}{np}\PY{o}{.}\PY{n}{dot}\PY{p}{(}\PY{n}{A}\PY{p}{,}\PY{n}{B}\PY{p}{)}
          \PY{n+nb}{print}\PY{p}{(}\PY{n}{C}\PY{p}{)}
\end{Verbatim}


    \begin{Verbatim}[commandchars=\\\{\}]
[140 320 500]

    \end{Verbatim}

    As an example, if \(\mathbf{B}\) were a row vector:

\begin{equation*}
\underbrace{
\begin{bmatrix}
1 & 2 & 3 \\
4 & 5 & 6 \\
7 & 8 & 9 \\
\end{bmatrix}
}_{\mathbf{A} \text{ 3 rows} \text{ 3 columns}}
\cdot
\underbrace{
\begin{bmatrix}
10 & 20 & 30 \\
\end{bmatrix}
}_{\mathbf{B} \text{  1 row} \text{  3 columns}}
\end{equation*}

We \emph{cannot} find the dot product \(\mathbf{B}\cdot\mathbf{A}\). The
number of columns in \(\mathbf{A}\) \textbf{is not} the same as number
of rows in \(\mathbf{B}\).

    We can swap the order of \(\mathbf{A}\) and \(\mathbf{B}\). The
multiplication is now possible. However, the outcome is different.

\begin{equation*}
\underbrace{
\begin{bmatrix}
10 & 20 & 30 \\
\end{bmatrix}
}_{\mathbf{B} \text{ 1 row} \text{ 3 columns}}
\cdot
\underbrace{
\begin{bmatrix}
\color{red}1 & \color{blue}2 & \color{green}3 \\
\color{red}4 & \color{blue}5 & \color{green}6 \\
\color{red}7 & \color{blue}8 & \color{green}9 \\
\end{bmatrix}
}_{\mathbf{A} \text{  3 rows} \text{  3 columns}}
=\underbrace{
\begin{bmatrix}
\color{red}{10 \cdot 1 + 20 \cdot 4 + 30 \cdot 7} &
\color{blue}{4 \cdot 10 + 5 \cdot 20 + 6 \cdot 30} &
\color{green}{7 \cdot 10 + 8 \cdot 20 + 9 \cdot 30} \\
\end{bmatrix}
}_{\mathbf{C} \text{  1 row} \text{  3 columns}}
=\underbrace{
\begin{bmatrix}
\color{red}{140} &
\color{blue}{320} &
\color{green}{500} \\
\end{bmatrix}
}_{\mathbf{C} \text{  3 rows} \text{  1 column1}}
\end{equation*}

    In Python, normal matrix multiplication rules apply to 2D arrays. This
holds even if the length of one of the dimensions of the 2D array is
equal to 1.

    \begin{Verbatim}[commandchars=\\\{\}]
{\color{incolor}In [{\color{incolor}249}]:} \PY{n}{A} \PY{o}{=} \PY{n}{np}\PY{o}{.}\PY{n}{array}\PY{p}{(}\PY{p}{[}\PY{p}{[}\PY{l+m+mi}{1}\PY{p}{,} \PY{l+m+mi}{2}\PY{p}{,} \PY{l+m+mi}{3}\PY{p}{]}\PY{p}{,} 
                        \PY{p}{[}\PY{l+m+mi}{4}\PY{p}{,} \PY{l+m+mi}{5}\PY{p}{,} \PY{l+m+mi}{6}\PY{p}{]}\PY{p}{,}
                        \PY{p}{[}\PY{l+m+mi}{7}\PY{p}{,} \PY{l+m+mi}{8}\PY{p}{,} \PY{l+m+mi}{9}\PY{p}{]}\PY{p}{]}\PY{p}{)}
          
          \PY{c+c1}{\PYZsh{} 2D array}
          \PY{n}{X} \PY{o}{=} \PY{n}{np}\PY{o}{.}\PY{n}{array}\PY{p}{(}\PY{p}{[}\PY{p}{[}\PY{l+m+mi}{10}\PY{p}{,} \PY{l+m+mi}{20}\PY{p}{,} \PY{l+m+mi}{30}\PY{p}{]}\PY{p}{]}\PY{p}{)}
          
          \PY{c+c1}{\PYZsh{} 2D array}
          \PY{n}{Y} \PY{o}{=} \PY{n}{np}\PY{o}{.}\PY{n}{array}\PY{p}{(}\PY{p}{[}\PY{p}{[}\PY{l+m+mi}{10}\PY{p}{]}\PY{p}{,}
                        \PY{p}{[}\PY{l+m+mi}{20}\PY{p}{]}\PY{p}{,} 
                        \PY{p}{[}\PY{l+m+mi}{30}\PY{p}{]}\PY{p}{]}\PY{p}{)}
          
          \PY{n+nb}{print}\PY{p}{(}\PY{n}{np}\PY{o}{.}\PY{n}{dot}\PY{p}{(}\PY{n}{X}\PY{p}{,} \PY{n}{A}\PY{p}{)}\PY{p}{)} \PY{c+c1}{\PYZsh{}, print(np.dot(A, X))}
          \PY{n+nb}{print}\PY{p}{(}\PY{n}{np}\PY{o}{.}\PY{n}{dot}\PY{p}{(}\PY{n}{A}\PY{p}{,} \PY{n}{Y}\PY{p}{)}\PY{p}{)} \PY{c+c1}{\PYZsh{}, print(np.dot(Y, A))}
\end{Verbatim}


    \begin{Verbatim}[commandchars=\\\{\}]
[[300 360 420]]
[[140]
 [320]
 [500]]

    \end{Verbatim}

    However, the orientation with which 1D arrays are shown (always
horizontal) does not impact their allowbale placement in an expression.

Python will automatially treat the 1D as a column where appropriate.

    \begin{Verbatim}[commandchars=\\\{\}]
{\color{incolor}In [{\color{incolor}250}]:} \PY{n}{A} \PY{o}{=} \PY{n}{np}\PY{o}{.}\PY{n}{array}\PY{p}{(}\PY{p}{[}\PY{p}{[}\PY{l+m+mi}{1}\PY{p}{,} \PY{l+m+mi}{2}\PY{p}{,} \PY{l+m+mi}{3}\PY{p}{]}\PY{p}{,} 
                        \PY{p}{[}\PY{l+m+mi}{4}\PY{p}{,} \PY{l+m+mi}{5}\PY{p}{,} \PY{l+m+mi}{6}\PY{p}{]}\PY{p}{,}
                        \PY{p}{[}\PY{l+m+mi}{7}\PY{p}{,} \PY{l+m+mi}{8}\PY{p}{,} \PY{l+m+mi}{9}\PY{p}{]}\PY{p}{]}\PY{p}{)}
          
          \PY{c+c1}{\PYZsh{} 1D array}
          \PY{n}{Z} \PY{o}{=} \PY{n}{np}\PY{o}{.}\PY{n}{array}\PY{p}{(}\PY{p}{[}\PY{l+m+mi}{10}\PY{p}{,} \PY{l+m+mi}{20}\PY{p}{,} \PY{l+m+mi}{30}\PY{p}{]}\PY{p}{)}
          
          \PY{n+nb}{print}\PY{p}{(}\PY{n}{np}\PY{o}{.}\PY{n}{dot}\PY{p}{(}\PY{n}{Z}\PY{p}{,} \PY{n}{A}\PY{p}{)}\PY{p}{)}
          \PY{n+nb}{print}\PY{p}{(}\PY{n}{np}\PY{o}{.}\PY{n}{dot}\PY{p}{(}\PY{n}{A}\PY{p}{,} \PY{n}{Z}\PY{p}{)}\PY{p}{)}
\end{Verbatim}


    \begin{Verbatim}[commandchars=\\\{\}]
[300 360 420]
[140 320 500]

    \end{Verbatim}

    \subsection{Useful Matrix Operations}\label{useful-matrix-operations}

    \paragraph{Inverse of a square matrix}\label{inverse-of-a-square-matrix}

    \begin{Verbatim}[commandchars=\\\{\}]
{\color{incolor}In [{\color{incolor}251}]:} \PY{n}{A} \PY{o}{=} \PY{n}{np}\PY{o}{.}\PY{n}{array}\PY{p}{(}\PY{p}{[}\PY{p}{[}\PY{l+m+mi}{1}\PY{p}{,}\PY{l+m+mi}{2}\PY{p}{]}\PY{p}{,} 
                        \PY{p}{[}\PY{l+m+mi}{3}\PY{p}{,} \PY{l+m+mi}{4}\PY{p}{]}\PY{p}{]}\PY{p}{)} 
          
          \PY{n}{Ainv} \PY{o}{=} \PY{n}{np}\PY{o}{.}\PY{n}{linalg}\PY{o}{.}\PY{n}{inv}\PY{p}{(}\PY{n}{A}\PY{p}{)}
          
          \PY{n+nb}{print}\PY{p}{(}\PY{n}{f}\PY{l+s+s2}{\PYZdq{}}\PY{l+s+s2}{A = }\PY{l+s+se}{\PYZbs{}n}\PY{l+s+s2}{ }\PY{l+s+si}{\PYZob{}A\PYZcb{}}\PY{l+s+s2}{\PYZdq{}}\PY{p}{)}
          \PY{n+nb}{print}\PY{p}{(}\PY{n}{f}\PY{l+s+s2}{\PYZdq{}}\PY{l+s+s2}{Inverse of A = }\PY{l+s+se}{\PYZbs{}n}\PY{l+s+s2}{ }\PY{l+s+si}{\PYZob{}Ainv\PYZcb{}}\PY{l+s+s2}{\PYZdq{}}\PY{p}{)}
\end{Verbatim}


    \begin{Verbatim}[commandchars=\\\{\}]
A = 
 [[1 2]
 [3 4]]
Inverse of A = 
 [[-2.   1. ]
 [ 1.5 -0.5]]

    \end{Verbatim}

    \paragraph{Determinant of a square
matrix}\label{determinant-of-a-square-matrix}

    \begin{Verbatim}[commandchars=\\\{\}]
{\color{incolor}In [{\color{incolor}252}]:} \PY{n}{A} \PY{o}{=} \PY{n}{np}\PY{o}{.}\PY{n}{array}\PY{p}{(}\PY{p}{[}\PY{p}{[}\PY{l+m+mi}{1}\PY{p}{,}\PY{l+m+mi}{2}\PY{p}{]}\PY{p}{,} 
                        \PY{p}{[}\PY{l+m+mi}{3}\PY{p}{,} \PY{l+m+mi}{4}\PY{p}{]}\PY{p}{]}\PY{p}{)} 
          
          \PY{n}{Adet} \PY{o}{=} \PY{n}{np}\PY{o}{.}\PY{n}{linalg}\PY{o}{.}\PY{n}{det}\PY{p}{(}\PY{n}{A}\PY{p}{)}
          
          \PY{n+nb}{print}\PY{p}{(}\PY{n}{f}\PY{l+s+s2}{\PYZdq{}}\PY{l+s+s2}{A = }\PY{l+s+se}{\PYZbs{}n}\PY{l+s+s2}{ }\PY{l+s+si}{\PYZob{}A\PYZcb{}}\PY{l+s+s2}{\PYZdq{}}\PY{p}{)}
          \PY{n+nb}{print}\PY{p}{(}\PY{n}{f}\PY{l+s+s2}{\PYZdq{}}\PY{l+s+s2}{Determinant of A = }\PY{l+s+s2}{\PYZob{}}\PY{l+s+s2}{round(Adet, 2)\PYZcb{}}\PY{l+s+s2}{\PYZdq{}}\PY{p}{)}
\end{Verbatim}


    \begin{Verbatim}[commandchars=\\\{\}]
A = 
 [[1 2]
 [3 4]]
Determinant of A = -2.0

    \end{Verbatim}

    \paragraph{Transpose of a matrix}\label{transpose-of-a-matrix}

\begin{itemize}
\tightlist
\item
  The columns of the transpose matrix are the rows of the original
  matrix.
\item
  The rows of the transopse matrix are the columns of the original
  matrix.
\end{itemize}

    \begin{Verbatim}[commandchars=\\\{\}]
{\color{incolor}In [{\color{incolor}253}]:} \PY{n}{a} \PY{o}{=} \PY{n}{np}\PY{o}{.}\PY{n}{zeros}\PY{p}{(}\PY{p}{(}\PY{l+m+mi}{2}\PY{p}{,}\PY{l+m+mi}{4}\PY{p}{)}\PY{p}{)}
          \PY{n+nb}{print}\PY{p}{(}\PY{n}{a}\PY{p}{)}
          \PY{n+nb}{print}\PY{p}{(}\PY{p}{)}
          
          
          \PY{n+nb}{print}\PY{p}{(}\PY{n}{a}\PY{o}{.}\PY{n}{T}\PY{p}{)}
          \PY{n+nb}{print}\PY{p}{(}\PY{p}{)}
          
          \PY{c+c1}{\PYZsh{}or }
          
          \PY{n+nb}{print}\PY{p}{(}\PY{n}{np}\PY{o}{.}\PY{n}{transpose}\PY{p}{(}\PY{n}{a}\PY{p}{)}\PY{p}{)}
\end{Verbatim}


    \begin{Verbatim}[commandchars=\\\{\}]
[[0. 0. 0. 0.]
 [0. 0. 0. 0.]]

[[0. 0.]
 [0. 0.]
 [0. 0.]
 [0. 0.]]

[[0. 0.]
 [0. 0.]
 [0. 0.]
 [0. 0.]]

    \end{Verbatim}

    \paragraph{Generate Identity Matrix}\label{generate-identity-matrix}

    \begin{Verbatim}[commandchars=\\\{\}]
{\color{incolor}In [{\color{incolor}254}]:} \PY{n}{I} \PY{o}{=} \PY{n}{np}\PY{o}{.}\PY{n}{eye}\PY{p}{(}\PY{l+m+mi}{2}\PY{p}{)}
          \PY{n+nb}{print}\PY{p}{(}\PY{n}{I}\PY{p}{)}
          
          \PY{n+nb}{print}\PY{p}{(}\PY{p}{)}
          
          \PY{n}{I} \PY{o}{=} \PY{n}{np}\PY{o}{.}\PY{n}{eye}\PY{p}{(}\PY{l+m+mi}{4}\PY{p}{)}
          \PY{n+nb}{print}\PY{p}{(}\PY{n}{I}\PY{p}{)}
\end{Verbatim}


    \begin{Verbatim}[commandchars=\\\{\}]
[[1. 0.]
 [0. 1.]]

[[1. 0. 0. 0.]
 [0. 1. 0. 0.]
 [0. 0. 1. 0.]
 [0. 0. 0. 1.]]

    \end{Verbatim}

    \subsection{Systems of Ordinary Differential
Equations}\label{systems-of-ordinary-differential-equations}

 \texttt{odeint} can also be used to solve \emph{systems} of coupled
differential equations.

    \subsubsection{Example : Migration
problem}\label{example-migration-problem}

A well known problem concerns an imaginary country with three cities, A,
B and C. At the end of each year, a fraction, \(n\) of the people must
leave each city. Half of the people leaving a city move to one of the
two options, and half to the other.

    This gives us a system of simultaneous equations

\begin{align*}
\Delta A = \frac{Bn}{2} + \frac{Cn}{2} - An \\
\Delta B = \frac{An}{2} + \frac{Cn}{2} - Bn\\
\Delta C = \frac{An}{2} + \frac{Bn}{2} -Cn \\
\end{align*}

    Matrices are a convenient way to represent this problem.

    \begin{equation*}
\begin{bmatrix}
\Delta A \\
\Delta B \\
\Delta C \\
\end{bmatrix}
=
\underbrace{
\begin{bmatrix}
-n & \frac{n}{2} & \frac{n}{2} \\
\frac{n}{2} & -n & \frac{n}{2} \\
\frac{n}{2} & \frac{n}{2} & -n \\
\end{bmatrix}
}_{\mathbf{migration}}
\cdot
\underbrace{
\begin{bmatrix}
A \\
B \\
C \\
\end{bmatrix}
}_{\mathbf{population}}
\end{equation*}

    Let's use variable names \(\mathbf{M}\) and \(\mathbf{P}\).

    \begin{equation*}
\begin{bmatrix}
\Delta a \\
\Delta b \\
\Delta c \\
\end{bmatrix}
=
\underbrace{
\begin{bmatrix}
-0.2 & 0.1 & 0.1 \\
0.1 & -0.2 & 0.1 \\
0.1 & 0.1 & -0.2 \\
\end{bmatrix}
}_{\mathbf{M}}
\cdot
\underbrace{
\begin{bmatrix}
A \\
B \\
C \\
\end{bmatrix}
}_{\mathbf{P}}
\end{equation*}

    We can use \texttt{odeint} to solve coupled first order ordinary
differential equations simultaneously (systems of ODEs).

    First create matrix \(\mathbf{M}\):

    \begin{Verbatim}[commandchars=\\\{\}]
{\color{incolor}In [{\color{incolor}255}]:} \PY{k+kn}{import} \PY{n+nn}{numpy} \PY{k}{as} \PY{n+nn}{np}
          \PY{n}{M} \PY{o}{=} \PY{n}{np}\PY{o}{.}\PY{n}{full}\PY{p}{(}\PY{p}{(}\PY{l+m+mi}{3}\PY{p}{,} \PY{l+m+mi}{3}\PY{p}{)}\PY{p}{,} \PY{l+m+mf}{0.01}\PY{p}{)}
          \PY{n}{np}\PY{o}{.}\PY{n}{fill\PYZus{}diagonal}\PY{p}{(}\PY{n}{M}\PY{p}{,} \PY{o}{\PYZhy{}}\PY{l+m+mf}{0.02}\PY{p}{)}
          \PY{n+nb}{print}\PY{p}{(}\PY{n}{M}\PY{p}{)}
\end{Verbatim}


    \begin{Verbatim}[commandchars=\\\{\}]
[[-0.02  0.01  0.01]
 [ 0.01 -0.02  0.01]
 [ 0.01  0.01 -0.02]]

    \end{Verbatim}

    The initial value variable is a list.

Note, the values are input as a single data structure.

The elements are the initial populations \(A\), \(B\), and \(C\).

    \begin{Verbatim}[commandchars=\\\{\}]
{\color{incolor}In [{\color{incolor}256}]:} \PY{c+c1}{\PYZsh{} Initial population}
          \PY{n}{P0} \PY{o}{=} \PY{n}{np}\PY{o}{.}\PY{n}{array}\PY{p}{(}\PY{p}{[}\PY{l+m+mf}{190.0}\PY{p}{,} \PY{l+m+mf}{500.0}\PY{p}{,} \PY{l+m+mf}{30.0}\PY{p}{]}\PY{p}{)}
\end{Verbatim}


    Timespan over which to evaluate the change in population:

    \begin{Verbatim}[commandchars=\\\{\}]
{\color{incolor}In [{\color{incolor}257}]:} \PY{c+c1}{\PYZsh{} Time steps to evaluate}
          \PY{n}{tspan} \PY{o}{=} \PY{n}{np}\PY{o}{.}\PY{n}{arange}\PY{p}{(}\PY{l+m+mi}{0}\PY{p}{,} \PY{l+m+mi}{150}\PY{p}{)}
\end{Verbatim}


    The chnage in population is the dot product (or matrix product) of
matrix \(\mathbf{M}\), and the current populations, matrix
\(\mathbf{P}\).

Therefore the inputs to the function are: - current population (matrix
\(\mathbf{P}\)) - timespan to investigate - matrix \(\mathbf{M}\)

The function should output the \textbf{rate of change of each variable}
as a \textbf{single list}.

    \begin{Verbatim}[commandchars=\\\{\}]
{\color{incolor}In [{\color{incolor}258}]:} \PY{k}{def} \PY{n+nf}{dP\PYZus{}dt}\PY{p}{(}\PY{n}{P}\PY{p}{,} \PY{n}{t}\PY{p}{,} \PY{n}{M}\PY{p}{)}\PY{p}{:}
              
              \PY{n}{dP\PYZus{}dt} \PY{o}{=} \PY{n}{np}\PY{o}{.}\PY{n}{dot}\PY{p}{(}\PY{n}{P}\PY{p}{,} \PY{n}{M}\PY{p}{)}
              
              \PY{k}{return} \PY{p}{[}\PY{n}{dP\PYZus{}dt}\PY{p}{[}\PY{l+m+mi}{0}\PY{p}{]}\PY{p}{,} 
                      \PY{n}{dP\PYZus{}dt}\PY{p}{[}\PY{l+m+mi}{1}\PY{p}{]}\PY{p}{,} 
                      \PY{n}{dP\PYZus{}dt}\PY{p}{[}\PY{l+m+mi}{2}\PY{p}{]}\PY{p}{]}
\end{Verbatim}


    We run the \texttt{odeint} solver as before. Inputs: - function -
initial condition - timespan - additional arguments to function (matrix
\(\mathbf{M}\))

    \begin{Verbatim}[commandchars=\\\{\}]
{\color{incolor}In [{\color{incolor}259}]:} \PY{n}{Ps} \PY{o}{=} \PY{n}{odeint}\PY{p}{(}\PY{n}{dP\PYZus{}dt}\PY{p}{,} \PY{n}{P0}\PY{p}{,} \PY{n}{tspan}\PY{p}{,} \PY{n}{args}\PY{o}{=}\PY{p}{(}\PY{n}{M}\PY{p}{,}\PY{p}{)}\PY{p}{)}
\end{Verbatim}


    As before, the output is retured as a vector, each time, there is one
vector for each population.

    \begin{Verbatim}[commandchars=\\\{\}]
{\color{incolor}In [{\color{incolor}260}]:} \PY{c+c1}{\PYZsh{} first 5 rows of Ps}
          \PY{n+nb}{print}\PY{p}{(}\PY{n}{Ps}\PY{p}{[}\PY{p}{:}\PY{l+m+mi}{5}\PY{p}{,} \PY{p}{:}\PY{p}{]}\PY{p}{)}
\end{Verbatim}


    \begin{Verbatim}[commandchars=\\\{\}]
[[190.         500.          30.        ]
 [191.47772343 492.31583817  36.2064384 ]
 [192.91177343 484.85877818  42.22944839]
 [194.30344084 477.62210765  48.07445151]
 [195.65397826 470.59931305  53.74670869]]

    \end{Verbatim}

    \begin{Verbatim}[commandchars=\\\{\}]
{\color{incolor}In [{\color{incolor}261}]:} \PY{n}{plt}\PY{o}{.}\PY{n}{plot}\PY{p}{(}\PY{n}{tspan}\PY{p}{,} \PY{n}{Ps}\PY{p}{[}\PY{p}{:}\PY{p}{,}\PY{l+m+mi}{0}\PY{p}{]}\PY{p}{,} \PY{n}{label}\PY{o}{=}\PY{l+s+s2}{\PYZdq{}}\PY{l+s+s2}{A}\PY{l+s+s2}{\PYZdq{}}\PY{p}{)}
          \PY{n}{plt}\PY{o}{.}\PY{n}{plot}\PY{p}{(}\PY{n}{tspan}\PY{p}{,} \PY{n}{Ps}\PY{p}{[}\PY{p}{:}\PY{p}{,}\PY{l+m+mi}{1}\PY{p}{]}\PY{p}{,} \PY{n}{label}\PY{o}{=}\PY{l+s+s2}{\PYZdq{}}\PY{l+s+s2}{B}\PY{l+s+s2}{\PYZdq{}}\PY{p}{)}
          \PY{n}{plt}\PY{o}{.}\PY{n}{plot}\PY{p}{(}\PY{n}{tspan}\PY{p}{,} \PY{n}{Ps}\PY{p}{[}\PY{p}{:}\PY{p}{,}\PY{l+m+mi}{2}\PY{p}{]}\PY{p}{,} \PY{n}{label}\PY{o}{=}\PY{l+s+s2}{\PYZdq{}}\PY{l+s+s2}{C}\PY{l+s+s2}{\PYZdq{}}\PY{p}{)}
          \PY{n}{plt}\PY{o}{.}\PY{n}{xlabel}\PY{p}{(}\PY{l+s+s2}{\PYZdq{}}\PY{l+s+s2}{Time}\PY{l+s+s2}{\PYZdq{}}\PY{p}{)}
          \PY{n}{plt}\PY{o}{.}\PY{n}{ylabel}\PY{p}{(}\PY{l+s+s2}{\PYZdq{}}\PY{l+s+s2}{Population}\PY{l+s+s2}{\PYZdq{}}\PY{p}{)}
          \PY{n}{plt}\PY{o}{.}\PY{n}{legend}\PY{p}{(}\PY{p}{)}\PY{p}{;}
\end{Verbatim}


    \begin{center}
    \adjustimage{max size={0.9\linewidth}{0.9\paperheight}}{output_99_0.png}
    \end{center}
    { \hspace*{\fill} \\}
    
    For example we can use \texttt{odeint} to solve the population problem
from earlier.

The function dP\_dt is exactly the same as the function
\texttt{change\_pop} that we used earlier. The only difference is that
\texttt{dP\_dt} returns the individual rates of chnage of each
population as a list.

    We can check the solution by iteratively solving the equations over each
time step using a \texttt{for} loop:

    \begin{Verbatim}[commandchars=\\\{\}]
{\color{incolor}In [{\color{incolor}262}]:} \PY{k}{def} \PY{n+nf}{dP\PYZus{}dt\PYZus{}manual}\PY{p}{(}\PY{n}{P}\PY{p}{,} \PY{n}{t}\PY{p}{,} \PY{n}{M}\PY{p}{)}\PY{p}{:}
              \PY{l+s+s2}{\PYZdq{}}\PY{l+s+s2}{Returns a vector of the population change over time t}\PY{l+s+s2}{\PYZdq{}}
              
              \PY{c+c1}{\PYZsh{} make 2D array to store population per timestep}
              \PY{n}{dPdt} \PY{o}{=} \PY{n}{P}
              \PY{k}{for} \PY{n}{time} \PY{o+ow}{in} \PY{n}{t}\PY{p}{[}\PY{p}{:}\PY{o}{\PYZhy{}}\PY{l+m+mi}{1}\PY{p}{]}\PY{p}{:}
                  
                  \PY{c+c1}{\PYZsh{} Increment population }
                  \PY{n}{P} \PY{o}{=} \PY{n}{P} \PY{o}{+} \PY{n}{np}\PY{o}{.}\PY{n}{dot}\PY{p}{(}\PY{n}{P}\PY{p}{,} \PY{n}{M}\PY{p}{)}
                  
                  \PY{c+c1}{\PYZsh{} Store population}
                  \PY{n}{dPdt} \PY{o}{=} \PY{n}{np}\PY{o}{.}\PY{n}{vstack}\PY{p}{(}\PY{p}{(}\PY{n}{dPdt}\PY{p}{,} \PY{n}{P}\PY{p}{)}\PY{p}{)}
                  
              \PY{k}{return} \PY{n}{dPdt}
                  
          \PY{n}{dPdt} \PY{o}{=} \PY{n}{dP\PYZus{}dt\PYZus{}manual}\PY{p}{(}\PY{n}{P0}\PY{p}{,} \PY{n}{tspan}\PY{p}{,} \PY{n}{M}\PY{p}{)} 
\end{Verbatim}


    Now we can plot the manually computed solution on top of the numerical
approximation using \texttt{odeint}:

    \begin{Verbatim}[commandchars=\\\{\}]
{\color{incolor}In [{\color{incolor}263}]:} \PY{c+c1}{\PYZsh{} Numerical Appproximation}
          \PY{n}{plt}\PY{o}{.}\PY{n}{plot}\PY{p}{(}\PY{n}{tspan}\PY{p}{,} \PY{n}{Ps}\PY{p}{[}\PY{p}{:}\PY{p}{,}\PY{l+m+mi}{0}\PY{p}{]}\PY{p}{,} \PY{n}{label}\PY{o}{=}\PY{l+s+s2}{\PYZdq{}}\PY{l+s+s2}{A}\PY{l+s+s2}{\PYZdq{}}\PY{p}{)}
          \PY{n}{plt}\PY{o}{.}\PY{n}{plot}\PY{p}{(}\PY{n}{tspan}\PY{p}{,} \PY{n}{Ps}\PY{p}{[}\PY{p}{:}\PY{p}{,}\PY{l+m+mi}{1}\PY{p}{]}\PY{p}{,} \PY{n}{label}\PY{o}{=}\PY{l+s+s2}{\PYZdq{}}\PY{l+s+s2}{B}\PY{l+s+s2}{\PYZdq{}}\PY{p}{)}
          \PY{n}{plt}\PY{o}{.}\PY{n}{plot}\PY{p}{(}\PY{n}{tspan}\PY{p}{,} \PY{n}{Ps}\PY{p}{[}\PY{p}{:}\PY{p}{,}\PY{l+m+mi}{2}\PY{p}{]}\PY{p}{,} \PY{n}{label}\PY{o}{=}\PY{l+s+s2}{\PYZdq{}}\PY{l+s+s2}{C}\PY{l+s+s2}{\PYZdq{}}\PY{p}{)}
          
          \PY{n}{plt}\PY{o}{.}\PY{n}{plot}\PY{p}{(}\PY{n}{tspan}\PY{p}{,} \PY{n}{dPdt}\PY{p}{[}\PY{p}{:}\PY{p}{,}\PY{l+m+mi}{0}\PY{p}{]}\PY{p}{,} \PY{l+s+s1}{\PYZsq{}}\PY{l+s+s1}{:}\PY{l+s+s1}{\PYZsq{}}\PY{p}{,} \PY{n}{label}\PY{o}{=}\PY{l+s+s2}{\PYZdq{}}\PY{l+s+s2}{A}\PY{l+s+s2}{\PYZdq{}}\PY{p}{)}
          \PY{n}{plt}\PY{o}{.}\PY{n}{plot}\PY{p}{(}\PY{n}{tspan}\PY{p}{,} \PY{n}{dPdt}\PY{p}{[}\PY{p}{:}\PY{p}{,}\PY{l+m+mi}{1}\PY{p}{]}\PY{p}{,} \PY{l+s+s1}{\PYZsq{}}\PY{l+s+s1}{:}\PY{l+s+s1}{\PYZsq{}}\PY{p}{,} \PY{n}{label}\PY{o}{=}\PY{l+s+s2}{\PYZdq{}}\PY{l+s+s2}{B}\PY{l+s+s2}{\PYZdq{}}\PY{p}{)}
          \PY{n}{plt}\PY{o}{.}\PY{n}{plot}\PY{p}{(}\PY{n}{tspan}\PY{p}{,} \PY{n}{dPdt}\PY{p}{[}\PY{p}{:}\PY{p}{,}\PY{l+m+mi}{2}\PY{p}{]}\PY{p}{,} \PY{l+s+s1}{\PYZsq{}}\PY{l+s+s1}{:}\PY{l+s+s1}{\PYZsq{}}\PY{p}{,} \PY{n}{label}\PY{o}{=}\PY{l+s+s2}{\PYZdq{}}\PY{l+s+s2}{C}\PY{l+s+s2}{\PYZdq{}}\PY{p}{)}
          
          \PY{n}{plt}\PY{o}{.}\PY{n}{xlabel}\PY{p}{(}\PY{l+s+s2}{\PYZdq{}}\PY{l+s+s2}{Years}\PY{l+s+s2}{\PYZdq{}}\PY{p}{)}
          \PY{n}{plt}\PY{o}{.}\PY{n}{ylabel}\PY{p}{(}\PY{l+s+s2}{\PYZdq{}}\PY{l+s+s2}{Population}\PY{l+s+s2}{\PYZdq{}}\PY{p}{)}
          \PY{n}{plt}\PY{o}{.}\PY{n}{legend}\PY{p}{(}\PY{p}{)}\PY{p}{;}
\end{Verbatim}


    \begin{center}
    \adjustimage{max size={0.9\linewidth}{0.9\paperheight}}{output_104_0.png}
    \end{center}
    { \hspace*{\fill} \\}
    
    \subsection{Interactive Plots}\label{interactive-plots}

 Interactive plots allow us to dynamically observe the influence that
changing different parameters has on an output.

To produce animated plots, we first need to install a program, FFmpeg,
for handling multimedia data.

Follow the instructions below to install ffmpeg:

    \subsubsection{Installing FFmpeg}\label{installing-ffmpeg}

    Installing FFmpeg on mac

Open a terminal. Copy and paste the following command into the terminal
to install homebrew (a linux-like package manager):
\textgreater{}\texttt{ruby\ -e\ "\$(curl\ -fsSL\ https://raw.githubusercontent.com/Homebrew/install/master/install)"}

Press 'Enter' to install and enter your password when prompted.

Copy and paste the following command into the terminal to install
FFmpeg: \textgreater{}\texttt{brew\ install\ ffmpeg}

To check ffmpeg has installed copy and paste the following command into
the terminal: \textgreater{}\texttt{ffmpeg\ -version}

If FFmpeg has installed a few lines of code will appear, starting with
the version number which will be something like:
\textgreater{}\texttt{ffmpeg\ version\ 3.4\ Copyright\ (c)\ 2000-2017}

    Installing FFmpeg on linux

Open a terminal. Copy and paste the following commands into the terminal
(one-by-one, pressing enter after one) to install ffmpeg:
\textgreater{}\texttt{sudo\ add-apt-repository\ ppa:kirillshkrogalev/ffmpeg-next\ sudo\ apt-get\ update\ sudo\ apt-get\ install\ ffmpeg}

To check ffmpeg has installed copy and paste the following command into
the terminal: \textgreater{}\texttt{ffmpeg\ -version}

If FFmpeg has installed a few lines of code will appear, starting with
the version number which will be something like:
\textgreater{}\texttt{ffmpeg\ version\ 3.4\ Copyright\ (c)\ 2000-2017}

    Installing FFmpeg on windows

Open a terminal.

To find out what version of windows you have, copy and paste the
following command into the terminal to see if your computer has a 32 or
64 bit CPU: \textgreater{}\texttt{wmic\ os\ get\ osarchitecture}

Go to ffmpeg.zeranoe.com/builds/ - Click the appropraite
\textbf{Architecture} for your computer. - Click Static \textbf{Linking}
- Leave \textbf{Version} as the default. - Click \textbf{Download Build}

Go to your computer's Downloads folder. - Right click on the newly
downloaded .zip folder (it's name will start with \texttt{ffmpeg}). -
Choose \textbf{Extract All} from the drop-down menu. - If given the
option to choose a location to extract the files to, choose your Program
Files folder.Otherwise, a non-.zip folder of the same name as the one
you downloaded will appear in the Downloads folder. Copy the folder to
your Program Files folder. - Change rename of the folder you just copied
into Program Files with the name: \texttt{ffmpeg}

Go back to the terminal and copy and paste the following command into
the terminal to add ffmpeg to the windows path:
\textgreater{}\texttt{PATH=C:\textbackslash{}Program\ Files\textbackslash{}ffmpeg\textbackslash{}bin;\%PATH\%}

To check ffmpeg has installed copy and paste the following command into
the terminal: \textgreater{}\texttt{ffmpeg\ -version}

If FFmpeg has installed a few lines of code will appear, starting with
the version number which will be something like:
\textgreater{}\texttt{ffmpeg\ version\ 3.4\ Copyright\ (c)\ 2000-2017}

    As a simple example, we will plot

\[
f(t) = t^{\alpha} \sin(\omega t)
\]

\begin{itemize}
\tightlist
\item
  independent variable, \(t\)
\item
  dependent variable, \(f(t)\)
\item
  input parameters, \(\alpha\) and \(\omega\)
\end{itemize}

We will create two sliders allowing us to change the value of \(\alpha\)
and \(\omega\) \emph{interactively}.

We can observe the resulting change in \(f(t)\) dynamically.

    Interactive plots be can created using Python module
\texttt{ipywidgets}.

Run the following cell to install \texttt{ipywidgets}

    \begin{Verbatim}[commandchars=\\\{\}]
{\color{incolor}In [{\color{incolor}264}]:} \PY{k}{try}\PY{p}{:}
              \PY{k+kn}{import} \PY{n+nn}{ipywidgets}
          \PY{k}{except} \PY{n+ne}{ImportError}\PY{p}{:}
              \PY{k}{try}\PY{p}{:}
                  \PY{o}{!}\PY{o}{\PYZob{}}sys.executable\PY{o}{\PYZcb{}} \PYZhy{}m pip \PYZhy{}q install ipywidgets
                  \PY{k+kn}{import} \PY{n+nn}{ipywidgets}
              \PY{k}{except} \PY{n+ne}{ImportError}\PY{p}{:}
                  \PY{o}{!}\PY{o}{\PYZob{}}sys.executable\PY{o}{\PYZcb{}} \PYZhy{}m pip \PYZhy{}q \PYZhy{}\PYZhy{}user install ipywidgets
              \PY{k}{finally}\PY{p}{:}
                  \PY{o}{!}jupyter nbextension \PY{n+nb}{enable} \PYZhy{}\PYZhy{}py widgetsnbextension
                  \PY{n+nb}{print}\PY{p}{(}\PY{l+s+s2}{\PYZdq{}}\PY{l+s+s2}{You will need to refresh your browser page}\PY{l+s+s2}{\PYZdq{}}\PY{p}{)}        
          \PY{k+kn}{from} \PY{n+nn}{ipywidgets} \PY{k}{import} \PY{n}{interact}
\end{Verbatim}


    \[
f(t) = t^{\alpha} \sin(\omega t)
\]

Let's assign arbitrary values \(\alpha = 0\) and \$\omega =1 \$

Start by plotting \(f(t)\) against \(t\) as usual.

\emph{(To generate symbols α or ω type \texttt{\textbackslash{}omega} or
\texttt{\textbackslash{}alpha}, pressing \texttt{Tab} key at the end.)}

    \begin{Verbatim}[commandchars=\\\{\}]
{\color{incolor}In [{\color{incolor}265}]:} \PY{n}{α}\PY{p}{,} \PY{n}{ω} \PY{o}{=} \PY{l+m+mi}{0}\PY{p}{,} \PY{l+m+mi}{1}
          \PY{n}{t} \PY{o}{=} \PY{n}{np}\PY{o}{.}\PY{n}{linspace}\PY{p}{(}\PY{l+m+mi}{0}\PY{p}{,} \PY{l+m+mi}{2}\PY{o}{*}\PY{n}{np}\PY{o}{.}\PY{n}{pi}\PY{p}{,} \PY{l+m+mi}{200}\PY{p}{)}  
          
          \PY{n}{plt}\PY{o}{.}\PY{n}{plot}\PY{p}{(}\PY{n}{t}\PY{p}{,} \PY{p}{(}\PY{n}{t}\PY{o}{*}\PY{o}{*}\PY{n}{α}\PY{p}{)}\PY{o}{*}\PY{n}{np}\PY{o}{.}\PY{n}{sin}\PY{p}{(}\PY{n}{ω}\PY{o}{*}\PY{n}{t}\PY{p}{)}\PY{p}{)}
          
          \PY{n}{plt}\PY{o}{.}\PY{n}{xlabel}\PY{p}{(}\PY{l+s+s1}{\PYZsq{}}\PY{l+s+s1}{\PYZdl{}t\PYZdl{}}\PY{l+s+s1}{\PYZsq{}}\PY{p}{)}
          \PY{n}{plt}\PY{o}{.}\PY{n}{ylabel}\PY{p}{(}\PY{l+s+s1}{\PYZsq{}}\PY{l+s+s1}{\PYZdl{}f\PYZdl{}}\PY{l+s+s1}{\PYZsq{}}\PY{p}{)}
          \PY{n}{plt}\PY{o}{.}\PY{n}{title}\PY{p}{(}\PY{n}{f}\PY{l+s+s2}{\PYZdq{}}\PY{l+s+s2}{\PYZdl{}α\PYZdl{} = }\PY{l+s+si}{\PYZob{}α\PYZcb{}}\PY{l+s+s2}{, \PYZdl{}ω\PYZdl{} = }\PY{l+s+si}{\PYZob{}ω\PYZcb{}}\PY{l+s+s2}{\PYZdq{}}\PY{p}{)}
\end{Verbatim}


\begin{Verbatim}[commandchars=\\\{\}]
{\color{outcolor}Out[{\color{outcolor}265}]:} Text(0.5,1,'\$α\$ = 0, \$ω\$ = 1')
\end{Verbatim}
            
    \begin{center}
    \adjustimage{max size={0.9\linewidth}{0.9\paperheight}}{output_114_1.png}
    \end{center}
    { \hspace*{\fill} \\}
    
    Next encasulate the code to generate the plot within a function.

The parameters we want to vary are given as function arguments.

The default values are the values used in the previous example.

    \begin{Verbatim}[commandchars=\\\{\}]
{\color{incolor}In [{\color{incolor}266}]:} \PY{k}{def} \PY{n+nf}{plot}\PY{p}{(}\PY{n}{α}\PY{o}{=}\PY{l+m+mi}{0}\PY{p}{,} \PY{n}{ω}\PY{o}{=}\PY{l+m+mi}{1}\PY{p}{)}\PY{p}{:}
              \PY{l+s+s2}{\PYZdq{}}\PY{l+s+s2}{A plot of the function f(t)= (t**α)*np.sin(ω*t)}\PY{l+s+s2}{\PYZdq{}}
              \PY{n}{t} \PY{o}{=} \PY{n}{np}\PY{o}{.}\PY{n}{linspace}\PY{p}{(}\PY{l+m+mi}{0}\PY{p}{,} \PY{l+m+mi}{2}\PY{o}{*}\PY{n}{np}\PY{o}{.}\PY{n}{pi}\PY{p}{,} \PY{l+m+mi}{200}\PY{p}{)}  
              
              \PY{n}{plt}\PY{o}{.}\PY{n}{plot}\PY{p}{(}\PY{n}{t}\PY{p}{,} \PY{p}{(}\PY{n}{t}\PY{o}{*}\PY{o}{*}\PY{n}{α}\PY{p}{)}\PY{o}{*}\PY{n}{np}\PY{o}{.}\PY{n}{sin}\PY{p}{(}\PY{n}{ω}\PY{o}{*}\PY{n}{t}\PY{p}{)}\PY{p}{)}
              
              \PY{n}{plt}\PY{o}{.}\PY{n}{xlabel}\PY{p}{(}\PY{l+s+s1}{\PYZsq{}}\PY{l+s+s1}{\PYZdl{}t\PYZdl{}}\PY{l+s+s1}{\PYZsq{}}\PY{p}{)}
              \PY{n}{plt}\PY{o}{.}\PY{n}{ylabel}\PY{p}{(}\PY{l+s+s1}{\PYZsq{}}\PY{l+s+s1}{\PYZdl{}f\PYZdl{}}\PY{l+s+s1}{\PYZsq{}}\PY{p}{)}
              \PY{n}{plt}\PY{o}{.}\PY{n}{title}\PY{p}{(}\PY{l+s+sa}{r}\PY{l+s+s2}{\PYZdq{}}\PY{l+s+s2}{\PYZdl{}}\PY{l+s+s2}{\PYZbs{}}\PY{l+s+s2}{alpha\PYZdl{} = }\PY{l+s+si}{\PYZob{}\PYZcb{}}\PY{l+s+s2}{, \PYZdl{}}\PY{l+s+s2}{\PYZbs{}}\PY{l+s+s2}{omega\PYZdl{} = }\PY{l+s+si}{\PYZob{}\PYZcb{}}\PY{l+s+s2}{\PYZdq{}}\PY{o}{.}\PY{n}{format}\PY{p}{(}\PY{n}{α}\PY{p}{,} \PY{n}{ω}\PY{p}{)}\PY{p}{)}
              
\end{Verbatim}


    The \texttt{interact} function, takes the following arguments: - the
plotting function - a tuple for each variable parameter: - maximum value
(inclusive) - minimum value (inclusive) - step size between each
possible value

    \begin{Verbatim}[commandchars=\\\{\}]
{\color{incolor}In [{\color{incolor}267}]:} \PY{n}{interact}\PY{p}{(}\PY{n}{plot}\PY{p}{,} \PY{n}{α}\PY{o}{=}\PY{p}{(}\PY{l+m+mi}{0}\PY{p}{,} \PY{l+m+mi}{2}\PY{p}{,} \PY{l+m+mf}{0.25}\PY{p}{)}\PY{p}{,} \PY{n}{ω}\PY{o}{=}\PY{p}{(}\PY{o}{\PYZhy{}}\PY{l+m+mi}{10}\PY{p}{,} \PY{l+m+mi}{10}\PY{p}{,} \PY{l+m+mf}{0.25}\PY{p}{)}\PY{p}{)}\PY{p}{;}
\end{Verbatim}


    
    \begin{verbatim}
interactive(children=(FloatSlider(value=0.0, description='α', max=2.0, step=0.25), FloatSlider(value=1.0, desc…
    \end{verbatim}

    
    You can now adjust the values by moving the sliders.

The new values appear as the title to the plot.

    \subsection{Animated Plots}\label{animated-plots}

 Plotting can be a very useful way to visualise what is happening in a
dynamic physical system.

Matplotlib can be used to create animated plots showing the change in a
system over time.

    We will use the \texttt{animation} and \texttt{rc} subpackages of
matplotlib.

We will also use \texttt{IPython.display.HTML} to view the animated
output within jupyter notebook.

    \begin{Verbatim}[commandchars=\\\{\}]
{\color{incolor}In [{\color{incolor}268}]:} \PY{k+kn}{from} \PY{n+nn}{matplotlib} \PY{k}{import} \PY{n}{animation}\PY{p}{,} \PY{n}{rc}
          \PY{k+kn}{from} \PY{n+nn}{IPython}\PY{n+nn}{.}\PY{n+nn}{display} \PY{k}{import} \PY{n}{HTML}
\end{Verbatim}


    Let's start with a simple example to learn how to build an animation.

We will create an animation of a physical model.

    The set of steps to build an animated plot is: 1. Create a figure window
1. Create axes within the window 1. Create object(s) to animate e.g. a
line or point 1. Define an animation function for the change you want to
see at each timestep 1. Use the function
\texttt{animation.FuncAnimation} to create your animation and give it a
name. 1. Call the animation name to play it. 1. (Save the animation)

    \subsubsection{Example: Animating a
Line}\label{example-animating-a-line}

\paragraph{An Animated Sine Wave.}\label{an-animated-sine-wave.}

    \begin{Verbatim}[commandchars=\\\{\}]
{\color{incolor}In [{\color{incolor}269}]:} \PY{k+kn}{import} \PY{n+nn}{matplotlib}\PY{n+nn}{.}\PY{n+nn}{pyplot} \PY{k}{as} \PY{n+nn}{plt}
          \PY{k+kn}{import} \PY{n+nn}{numpy} \PY{k}{as} \PY{n+nn}{np}
          \PY{c+c1}{\PYZsh{} 1. Create a figure window. }
          \PY{n}{fig} \PY{o}{=} \PY{n}{plt}\PY{o}{.}\PY{n}{figure}\PY{p}{(}\PY{p}{)}
          
          \PY{c+c1}{\PYZsh{} 2. Creates axes within the window}
          \PY{n}{ax} \PY{o}{=} \PY{n}{plt}\PY{o}{.}\PY{n}{axes}\PY{p}{(}\PY{n}{xlim}\PY{o}{=}\PY{p}{(}\PY{l+m+mi}{0}\PY{p}{,} \PY{l+m+mi}{2}\PY{p}{)}\PY{p}{,} \PY{n}{ylim}\PY{o}{=}\PY{p}{(}\PY{o}{\PYZhy{}}\PY{l+m+mi}{2}\PY{p}{,} \PY{l+m+mi}{2}\PY{p}{)}\PY{p}{)}
          
          \PY{c+c1}{\PYZsh{} 3. Empty object (no data opints) to animate e.g. a line }
          \PY{c+c1}{\PYZsh{} Name must end with a `,` comma.  }
          \PY{n}{line}\PY{p}{,} \PY{o}{=} \PY{n}{ax}\PY{o}{.}\PY{n}{plot}\PY{p}{(}\PY{p}{[}\PY{p}{]}\PY{p}{,} \PY{p}{[}\PY{p}{]}\PY{p}{,} \PY{n}{lw}\PY{o}{=}\PY{l+m+mi}{2}\PY{p}{)}
          
          \PY{c+c1}{\PYZsh{} 4. Animation function: called sequentially}
          \PY{c+c1}{\PYZsh{} i = frame number.}
          \PY{c+c1}{\PYZsh{} Sine wave generated, phase shift proportional to i}
          \PY{k}{def} \PY{n+nf}{animate}\PY{p}{(}\PY{n}{i}\PY{p}{)}\PY{p}{:}
              \PY{n}{x} \PY{o}{=} \PY{n}{np}\PY{o}{.}\PY{n}{linspace}\PY{p}{(}\PY{l+m+mi}{0}\PY{p}{,} \PY{l+m+mi}{2}\PY{p}{,} \PY{l+m+mi}{1000}\PY{p}{)}
              \PY{n}{y} \PY{o}{=} \PY{n}{np}\PY{o}{.}\PY{n}{sin}\PY{p}{(}\PY{l+m+mi}{2} \PY{o}{*} \PY{n}{np}\PY{o}{.}\PY{n}{pi} \PY{o}{*} \PY{p}{(}\PY{n}{x} \PY{o}{\PYZhy{}} \PY{l+m+mf}{0.01} \PY{o}{*} \PY{n}{i}\PY{p}{)}\PY{p}{)}
              \PY{n}{line}\PY{o}{.}\PY{n}{set\PYZus{}data}\PY{p}{(}\PY{n}{x}\PY{p}{,} \PY{n}{y}\PY{p}{)}
              \PY{c+c1}{\PYZsh{} single return arguments should be given as a tuple with one value}
              \PY{k}{return} \PY{p}{(}\PY{n}{line}\PY{p}{,}\PY{p}{)}
          
          \PY{c+c1}{\PYZsh{} 5. Animates the data; }
          \PY{c+c1}{\PYZsh{} 100 frames}
          \PY{c+c1}{\PYZsh{} 20ms delay between frames}
          \PY{n}{anim} \PY{o}{=} \PY{n}{animation}\PY{o}{.}\PY{n}{FuncAnimation}\PY{p}{(}\PY{n}{fig}\PY{p}{,} \PY{n}{animate}\PY{p}{,} \PY{n}{frames}\PY{o}{=}\PY{l+m+mi}{100}\PY{p}{,} \PY{n}{interval}\PY{o}{=}\PY{l+m+mi}{20}\PY{p}{)}
          
          \PY{c+c1}{\PYZsh{} Set the animation display format to html which the non\PYZhy{}Python parts of this notebook are written in.}
          \PY{n}{rc}\PY{p}{(}\PY{l+s+s1}{\PYZsq{}}\PY{l+s+s1}{animation}\PY{l+s+s1}{\PYZsq{}}\PY{p}{,} \PY{n}{html}\PY{o}{=}\PY{l+s+s1}{\PYZsq{}}\PY{l+s+s1}{html5}\PY{l+s+s1}{\PYZsq{}}\PY{p}{)}
          
          \PY{c+c1}{\PYZsh{} 6. Play the animation }
          \PY{n}{anim}
          
          \PY{c+c1}{\PYZsh{} 7. Save the animation as a .mp4 file \PYZhy{} if you save the file, it won\PYZsq{}t play}
          \PY{c+c1}{\PYZsh{} 15 frames per second}
          \PY{c+c1}{\PYZsh{} 1800 bits of data processed/stored per second}
          \PY{c+c1}{\PYZsh{}writer = animation.writers[\PYZsq{}ffmpeg\PYZsq{}](fps=15, bitrate=1800)}
          \PY{c+c1}{\PYZsh{}anim.save(\PYZsq{}img/sin\PYZus{}movie.mp4\PYZsq{}, writer=writer)}
\end{Verbatim}


\begin{Verbatim}[commandchars=\\\{\}]
{\color{outcolor}Out[{\color{outcolor}269}]:} <matplotlib.animation.FuncAnimation at 0x181f8fa390>
\end{Verbatim}
            
    \begin{center}
    \adjustimage{max size={0.9\linewidth}{0.9\paperheight}}{output_126_1.png}
    \end{center}
    { \hspace*{\fill} \\}
    
    When returning a single argument (e.g. \texttt{line}) it is returned as
a tuple with one value i.e. \texttt{(line,\ )}.

    \subsubsection{Example : Simulating Physical
Systems}\label{example-simulating-physical-systems}

\paragraph{Spring-Mass System}\label{spring-mass-system}

 If you are studying an engineering-related subject, you will most
likely study simple harmonic motion; a type of periodic motion or
oscillation motion.

For this oscillation to happen, the restoring force is: - directly
proportional to the displacement - in the direction opposite to the
displacement.

    A typical example of this is a mass attached to a spring.

    If we assume that: - the spring is ideal (it has no weight, mass, or
damping losses) - there is no friction

we can use a simple equation to give the position of the mass, \(x\), as
a function of time, \(t\):

\(x(t) = A cos(\omega t - \phi)\)

where: \(A\): Maximum amplitude (displacment from initial position),
defined by the initial conditions of the system. \(\phi\) : Phase (the
initial angle of a sinusoidal function at its origin) \(\omega\) :
Angular frequency (frequency of oscillation expressed in radians)

    Angular frequency

\(\omega=2\pi f = \sqrt{\frac{k}{m}}\)

where \(k\) : spring constant \(m\) : mass (kg) \(f\) : frequency (Hz)

    \begin{Verbatim}[commandchars=\\\{\}]
{\color{incolor}In [{\color{incolor}270}]:} \PY{c+c1}{\PYZsh{} 1. Create a figure window.}
          \PY{n}{fig} \PY{o}{=} \PY{n}{plt}\PY{o}{.}\PY{n}{figure}\PY{p}{(}\PY{p}{)}
          
          \PY{c+c1}{\PYZsh{} 2. Create axes within the window}
          \PY{n}{ax} \PY{o}{=} \PY{n}{plt}\PY{o}{.}\PY{n}{axes}\PY{p}{(}\PY{n}{xlim}\PY{o}{=}\PY{p}{(}\PY{o}{\PYZhy{}}\PY{l+m+mi}{2}\PY{p}{,} \PY{l+m+mi}{2}\PY{p}{)}\PY{p}{,} \PY{n}{ylim}\PY{o}{=}\PY{p}{(}\PY{o}{\PYZhy{}}\PY{l+m+mf}{3.5}\PY{p}{,}\PY{l+m+mf}{3.5}\PY{p}{)}\PY{p}{)}
          
          \PY{c+c1}{\PYZsh{} 3. Two objects to animate}
          \PY{n}{line}\PY{p}{,}   \PY{o}{=} \PY{n}{ax}\PY{o}{.}\PY{n}{plot}\PY{p}{(}\PY{p}{[}\PY{l+m+mi}{2}\PY{p}{,}\PY{l+m+mi}{1}\PY{p}{]}\PY{p}{,} \PY{p}{[}\PY{l+m+mi}{4}\PY{p}{,}\PY{l+m+mi}{3}\PY{p}{]}\PY{p}{,} \PY{n}{marker}\PY{o}{=}\PY{l+s+s2}{\PYZdq{}}\PY{l+s+s2}{\PYZdq{}} \PY{p}{,} \PY{n}{ls}\PY{o}{=}\PY{l+s+s2}{\PYZdq{}}\PY{l+s+s2}{\PYZhy{}}\PY{l+s+s2}{\PYZdq{}}\PY{p}{)}  \PY{c+c1}{\PYZsh{} a line}
          \PY{n}{point}\PY{p}{,} \PY{o}{=} \PY{n}{ax}\PY{o}{.}\PY{n}{plot}\PY{p}{(}\PY{p}{[}\PY{l+m+mi}{1}\PY{p}{]}\PY{p}{,} \PY{p}{[}\PY{l+m+mi}{1}\PY{p}{]}\PY{p}{,} \PY{n}{marker}\PY{o}{=}\PY{l+s+s1}{\PYZsq{}}\PY{l+s+s1}{o}\PY{l+s+s1}{\PYZsq{}}\PY{p}{,} \PY{n}{ms}\PY{o}{=}\PY{l+m+mi}{40}\PY{p}{)}        \PY{c+c1}{\PYZsh{} a point}
          
          \PY{c+c1}{\PYZsh{} Spring data}
          \PY{n}{k} \PY{o}{=} \PY{l+m+mi}{100}
          \PY{n}{m} \PY{o}{=} \PY{l+m+mi}{20}
          \PY{n}{w} \PY{o}{=} \PY{n}{np}\PY{o}{.}\PY{n}{sqrt}\PY{p}{(}\PY{n}{k}\PY{o}{/}\PY{n}{m}\PY{p}{)}
          \PY{n}{phi} \PY{o}{=} \PY{l+m+mi}{2}
          \PY{n}{A} \PY{o}{=} \PY{l+m+mi}{2}
          
          \PY{c+c1}{\PYZsh{} Position of mass as function of time}
          \PY{k}{def} \PY{n+nf}{fun}\PY{p}{(}\PY{n}{time}\PY{p}{)}\PY{p}{:}
              \PY{k}{return} \PY{n}{A}\PY{o}{*}\PY{n}{np}\PY{o}{.}\PY{n}{sin}\PY{p}{(}\PY{n}{w} \PY{o}{*} \PY{n}{time} \PY{o}{+} \PY{n}{phi}\PY{p}{)}
          
          \PY{c+c1}{\PYZsh{} 4. Animation function}
          \PY{c+c1}{\PYZsh{} Timestep = 1/20 = 0.05}
          \PY{k}{def} \PY{n+nf}{animate}\PY{p}{(}\PY{n}{t}\PY{p}{)}\PY{p}{:}    
              \PY{n}{x} \PY{o}{=} \PY{n}{fun}\PY{p}{(}\PY{n}{t}\PY{o}{/}\PY{l+m+mi}{20}\PY{p}{)}     
              \PY{n}{line}\PY{o}{.}\PY{n}{set\PYZus{}data}\PY{p}{(}\PY{p}{[}\PY{l+m+mi}{0}\PY{p}{,}\PY{l+m+mi}{0}\PY{p}{]}\PY{p}{,} \PY{p}{[}\PY{l+m+mi}{4}\PY{p}{,} \PY{o}{\PYZhy{}}\PY{n}{x}\PY{p}{]}\PY{p}{)}    
              \PY{n}{point}\PY{o}{.}\PY{n}{set\PYZus{}data}\PY{p}{(}\PY{l+m+mi}{0}\PY{p}{,} \PY{o}{\PYZhy{}}\PY{n}{x}\PY{p}{)}    
              \PY{k}{return} \PY{n}{line}\PY{p}{,} \PY{n}{point}
            
              
          \PY{c+c1}{\PYZsh{} 5. Create animation}
          \PY{c+c1}{\PYZsh{} 500 frames}
          \PY{c+c1}{\PYZsh{} 50ms (0.05s) delay between frames to match timestep}
          \PY{n}{anim} \PY{o}{=} \PY{n}{animation}\PY{o}{.}\PY{n}{FuncAnimation}\PY{p}{(}\PY{n}{fig}\PY{p}{,} \PY{n}{animate}\PY{p}{,} \PY{n}{frames}\PY{o}{=}\PY{l+m+mi}{500}\PY{p}{,} \PY{n}{interval}\PY{o}{=}\PY{l+m+mi}{50}\PY{p}{)}
          
          \PY{n}{rc}\PY{p}{(}\PY{l+s+s1}{\PYZsq{}}\PY{l+s+s1}{animation}\PY{l+s+s1}{\PYZsq{}}\PY{p}{,} \PY{n}{html}\PY{o}{=}\PY{l+s+s1}{\PYZsq{}}\PY{l+s+s1}{html5}\PY{l+s+s1}{\PYZsq{}}\PY{p}{)}
          
          \PY{c+c1}{\PYZsh{} 6. Play animation}
          \PY{n}{anim}
          
          \PY{c+c1}{\PYZsh{} 7. Save animation \PYZhy{} if uncommented, animation won\PYZsq{}t move}
          \PY{c+c1}{\PYZsh{} writer = animation.writers[\PYZsq{}ffmpeg\PYZsq{}](fps=15, bitrate=1800)}
          \PY{c+c1}{\PYZsh{} anim.save(\PYZsq{}img/spring\PYZus{}movie.mp4\PYZsq{}, writer=writer)}
\end{Verbatim}


\begin{Verbatim}[commandchars=\\\{\}]
{\color{outcolor}Out[{\color{outcolor}270}]:} <matplotlib.animation.FuncAnimation at 0x181fac8128>
\end{Verbatim}
            
    \begin{center}
    \adjustimage{max size={0.9\linewidth}{0.9\paperheight}}{output_132_1.png}
    \end{center}
    { \hspace*{\fill} \\}
    
    \subsubsection{Example : Simulating Physical
Systems}\label{example-simulating-physical-systems}

\paragraph{Trajectory : Plotting the
Object}\label{trajectory-plotting-the-object}

 A projectile is launched with initial velocity \(v\), at an angle of
\(\theta\).

If we neglect the force of drag on the projectile, it will travel in an
arc.

It's position in the horizontal (x) and vertical (y) direction at time
\(t\) can be found by.

\(x= vt\cos(\theta)\) \(y= vt\sin(\theta) - \frac{1}{2}gt^2\)

where gravity, \(g=9.81\)ms\(^{-2}\)

Initial conditions: - \(\theta = \pi/2\) - velocity, \(v=10\)ms\(^{-1}\)

    \begin{enumerate}
\def\labelenumi{\arabic{enumi}.}
\tightlist
\item
  Create a figure window
\item
  Create axes within the figure window.
\item
  Create a point to animate
\item
  Create variables for \(g\), \(v\), and \(\theta\)
\item
  Write a function, \texttt{fun} to find x and y position as a function
  of time, t (function argument = t)
\item
  To animate the motion of the particle create a function,
  \texttt{animate}. Within animate call the function, \texttt{func}. Use
  a timestep of i/10.
\item
  Use the function \texttt{animation.FuncAnimation} to create the
  animation.
\end{enumerate}

    \begin{Verbatim}[commandchars=\\\{\}]
{\color{incolor}In [{\color{incolor}271}]:} \PY{c+c1}{\PYZsh{} Creates a figure window.}
          \PY{n}{fig} \PY{o}{=} \PY{n}{plt}\PY{o}{.}\PY{n}{figure}\PY{p}{(}\PY{p}{)}
          
          \PY{c+c1}{\PYZsh{} Creates axes within the window}
          \PY{n}{ax} \PY{o}{=} \PY{n}{plt}\PY{o}{.}\PY{n}{axes}\PY{p}{(}\PY{n}{xlim}\PY{o}{=}\PY{p}{(}\PY{l+m+mi}{0}\PY{p}{,} \PY{l+m+mi}{10}\PY{p}{)}\PY{p}{,} \PY{n}{ylim}\PY{o}{=}\PY{p}{(}\PY{l+m+mi}{0}\PY{p}{,} \PY{l+m+mi}{5}\PY{p}{)}\PY{p}{)}
          
          \PY{c+c1}{\PYZsh{} Object to animate}
          \PY{n}{point}\PY{p}{,} \PY{o}{=} \PY{n}{ax}\PY{o}{.}\PY{n}{plot}\PY{p}{(}\PY{p}{[}\PY{l+m+mi}{1}\PY{p}{]}\PY{p}{,} \PY{p}{[}\PY{l+m+mi}{1}\PY{p}{]}\PY{p}{,} \PY{n}{marker}\PY{o}{=}\PY{l+s+s1}{\PYZsq{}}\PY{l+s+s1}{o}\PY{l+s+s1}{\PYZsq{}}\PY{p}{,} \PY{n}{ms}\PY{o}{=}\PY{l+m+mi}{40}\PY{p}{)}  \PY{c+c1}{\PYZsh{} for points}
          
          \PY{n}{g} \PY{o}{=} \PY{l+m+mf}{9.81}
          \PY{n}{v} \PY{o}{=} \PY{l+m+mi}{10}
          \PY{n}{theta} \PY{o}{=} \PY{n}{np}\PY{o}{.}\PY{n}{pi}\PY{o}{/}\PY{l+m+mi}{4}
          
          \PY{c+c1}{\PYZsh{} Position of mass as function of time}
          \PY{k}{def} \PY{n+nf}{fun}\PY{p}{(}\PY{n}{t}\PY{p}{)}\PY{p}{:}
              \PY{n}{x} \PY{o}{=} \PY{n}{v} \PY{o}{*} \PY{n}{t} \PY{o}{*} \PY{n}{np}\PY{o}{.}\PY{n}{cos}\PY{p}{(}\PY{n}{theta}\PY{p}{)}
              \PY{n}{y} \PY{o}{=} \PY{n}{v} \PY{o}{*} \PY{n}{t} \PY{o}{*} \PY{n}{np}\PY{o}{.}\PY{n}{sin}\PY{p}{(}\PY{n}{theta}\PY{p}{)} \PY{o}{\PYZhy{}} \PY{p}{(}\PY{l+m+mf}{0.5} \PY{o}{*} \PY{n}{g} \PY{o}{*} \PY{n}{t}\PY{o}{*}\PY{o}{*}\PY{l+m+mi}{2}\PY{p}{)}
              \PY{k}{return} \PY{n}{x}\PY{p}{,} \PY{n}{y}
          
          \PY{k}{def} \PY{n+nf}{animate}\PY{p}{(}\PY{n}{i}\PY{p}{)}\PY{p}{:}   
              \PY{n}{x}\PY{p}{,} \PY{n}{y} \PY{o}{=} \PY{n}{fun}\PY{p}{(}\PY{n}{i}\PY{o}{/}\PY{l+m+mi}{10}\PY{p}{)}   
              \PY{n}{point}\PY{o}{.}\PY{n}{set\PYZus{}data}\PY{p}{(}\PY{n}{x}\PY{p}{,} \PY{n}{y}\PY{p}{)}    
              \PY{k}{return}  \PY{p}{(}\PY{n}{point}\PY{p}{,}\PY{p}{)}
              
          \PY{c+c1}{\PYZsh{} Animates the data}
          \PY{c+c1}{\PYZsh{} 50 frames}
          \PY{c+c1}{\PYZsh{} 50ms delay between frames : animation plays at double speed of time\PYZhy{}varying system}
          \PY{n}{anim} \PY{o}{=} \PY{n}{animation}\PY{o}{.}\PY{n}{FuncAnimation}\PY{p}{(}\PY{n}{fig}\PY{p}{,} \PY{n}{animate}\PY{p}{,} \PY{n}{frames}\PY{o}{=}\PY{l+m+mi}{30}\PY{p}{,} \PY{n}{interval}\PY{o}{=}\PY{l+m+mi}{50}\PY{p}{,} \PY{n}{blit}\PY{o}{=}\PY{k+kc}{True}\PY{p}{)}
          
          \PY{n}{anim}
\end{Verbatim}


\begin{Verbatim}[commandchars=\\\{\}]
{\color{outcolor}Out[{\color{outcolor}271}]:} <matplotlib.animation.FuncAnimation at 0x181fac8080>
\end{Verbatim}
            
    \begin{center}
    \adjustimage{max size={0.9\linewidth}{0.9\paperheight}}{output_135_1.png}
    \end{center}
    { \hspace*{\fill} \\}
    
    \subsubsection{Example : Simulating Physical
Systems}\label{example-simulating-physical-systems}

\paragraph{Trajectory : Plotting the
Path}\label{trajectory-plotting-the-path}

    \begin{Verbatim}[commandchars=\\\{\}]
{\color{incolor}In [{\color{incolor}272}]:} \PY{k+kn}{import} \PY{n+nn}{matplotlib}\PY{n+nn}{.}\PY{n+nn}{pyplot} \PY{k}{as} \PY{n+nn}{plt}
          \PY{k+kn}{import} \PY{n+nn}{matplotlib}\PY{n+nn}{.}\PY{n+nn}{animation} \PY{k}{as} \PY{n+nn}{animation}
          \PY{k+kn}{import} \PY{n+nn}{time}
          
          \PY{n}{g} \PY{o}{=} \PY{l+m+mf}{9.81}
          \PY{n}{v} \PY{o}{=} \PY{l+m+mi}{10}
          \PY{n}{theta} \PY{o}{=} \PY{n}{np}\PY{o}{.}\PY{n}{pi}\PY{o}{/}\PY{l+m+mi}{4}
          
          \PY{n}{t} \PY{o}{=} \PY{n}{np}\PY{o}{.}\PY{n}{linspace}\PY{p}{(}\PY{l+m+mi}{0}\PY{p}{,} \PY{l+m+mi}{500}\PY{p}{)}
          \PY{n}{x} \PY{o}{=} \PY{n}{v} \PY{o}{*} \PY{n}{t} \PY{o}{*} \PY{n}{np}\PY{o}{.}\PY{n}{cos}\PY{p}{(}\PY{n}{theta}\PY{p}{)}
          \PY{n}{y} \PY{o}{=} \PY{n}{v} \PY{o}{*} \PY{n}{t} \PY{o}{*} \PY{n}{np}\PY{o}{.}\PY{n}{sin}\PY{p}{(}\PY{n}{theta}\PY{p}{)} \PY{o}{\PYZhy{}} \PY{p}{(}\PY{l+m+mf}{0.5} \PY{o}{*} \PY{n}{g} \PY{o}{*} \PY{n}{t}\PY{o}{*}\PY{o}{*}\PY{l+m+mi}{2}\PY{p}{)}
          
          \PY{n}{fig} \PY{o}{=} \PY{n}{plt}\PY{o}{.}\PY{n}{figure}\PY{p}{(}\PY{p}{)}
          \PY{n}{ax} \PY{o}{=} \PY{n}{plt}\PY{o}{.}\PY{n}{axes}\PY{p}{(}\PY{n}{xlim}\PY{o}{=}\PY{p}{(}\PY{l+m+mi}{0}\PY{p}{,} \PY{l+m+mi}{10}\PY{p}{)}\PY{p}{,} \PY{n}{ylim}\PY{o}{=}\PY{p}{(}\PY{l+m+mi}{0}\PY{p}{,} \PY{l+m+mi}{5}\PY{p}{)}\PY{p}{)}
          
          \PY{k}{def} \PY{n+nf}{animate}\PY{p}{(}\PY{n}{i}\PY{p}{)}\PY{p}{:}
          
              \PY{n}{i} \PY{o}{/}\PY{o}{=} \PY{l+m+mi}{10}
              \PY{k}{if} \PY{n}{i} \PY{o}{==} \PY{l+m+mi}{0}\PY{p}{:}
                  \PY{n}{s} \PY{o}{=} \PY{n}{i}
              \PY{k}{else}\PY{p}{:}
                  \PY{n}{s} \PY{o}{=} \PY{n}{np}\PY{o}{.}\PY{n}{linspace}\PY{p}{(}\PY{l+m+mi}{0}\PY{p}{,} \PY{n}{i}\PY{p}{)}
          
              \PY{n}{x} \PY{o}{=} \PY{n}{v} \PY{o}{*} \PY{n}{s} \PY{o}{*} \PY{n}{np}\PY{o}{.}\PY{n}{cos}\PY{p}{(}\PY{n}{theta}\PY{p}{)}
              \PY{n}{y} \PY{o}{=} \PY{n}{v} \PY{o}{*} \PY{n}{s} \PY{o}{*} \PY{n}{np}\PY{o}{.}\PY{n}{sin}\PY{p}{(}\PY{n}{theta}\PY{p}{)} \PY{o}{\PYZhy{}} \PY{p}{(}\PY{l+m+mf}{0.5} \PY{o}{*} \PY{n}{g} \PY{o}{*} \PY{n}{s}\PY{o}{*}\PY{o}{*}\PY{l+m+mi}{2}\PY{p}{)}
              \PY{n}{ax}\PY{o}{.}\PY{n}{plot}\PY{p}{(}\PY{n}{x}\PY{p}{,}\PY{n}{y}\PY{p}{,} \PY{l+s+s1}{\PYZsq{}}\PY{l+s+s1}{c}\PY{l+s+s1}{\PYZsq{}}\PY{p}{)}
              
          \PY{n}{ani} \PY{o}{=} \PY{n}{animation}\PY{o}{.}\PY{n}{FuncAnimation}\PY{p}{(}\PY{n}{fig}\PY{p}{,} \PY{n}{animate}\PY{p}{,} \PY{n}{frames}\PY{o}{=}\PY{l+m+mi}{15}\PY{p}{,} \PY{n}{interval}\PY{o}{=}\PY{l+m+mi}{50}\PY{p}{)}
          \PY{n}{ani}
\end{Verbatim}


\begin{Verbatim}[commandchars=\\\{\}]
{\color{outcolor}Out[{\color{outcolor}272}]:} <matplotlib.animation.FuncAnimation at 0x181f596a20>
\end{Verbatim}
            
    \begin{center}
    \adjustimage{max size={0.9\linewidth}{0.9\paperheight}}{output_137_1.png}
    \end{center}
    { \hspace*{\fill} \\}
    
    \subsubsection{Example : Simulating Physical
Systems}\label{example-simulating-physical-systems}

\paragraph{Trajectory : Plotting the Path \&
Object}\label{trajectory-plotting-the-path-object}

    \begin{Verbatim}[commandchars=\\\{\}]
{\color{incolor}In [{\color{incolor}273}]:} \PY{k+kn}{import} \PY{n+nn}{matplotlib}\PY{n+nn}{.}\PY{n+nn}{pyplot} \PY{k}{as} \PY{n+nn}{plt}
          \PY{k+kn}{import} \PY{n+nn}{matplotlib}\PY{n+nn}{.}\PY{n+nn}{animation} \PY{k}{as} \PY{n+nn}{animation}
          \PY{k+kn}{import} \PY{n+nn}{time}
          
          \PY{n}{g} \PY{o}{=} \PY{l+m+mf}{9.81}
          \PY{n}{v} \PY{o}{=} \PY{l+m+mi}{10}
          \PY{n}{theta} \PY{o}{=} \PY{n}{np}\PY{o}{.}\PY{n}{pi}\PY{o}{/}\PY{l+m+mi}{4}
          
          \PY{n}{t} \PY{o}{=} \PY{n}{np}\PY{o}{.}\PY{n}{linspace}\PY{p}{(}\PY{l+m+mi}{0}\PY{p}{,} \PY{l+m+mi}{500}\PY{p}{)}
          \PY{n}{x} \PY{o}{=} \PY{n}{v} \PY{o}{*} \PY{n}{t} \PY{o}{*} \PY{n}{np}\PY{o}{.}\PY{n}{cos}\PY{p}{(}\PY{n}{theta}\PY{p}{)}
          \PY{n}{y} \PY{o}{=} \PY{n}{v} \PY{o}{*} \PY{n}{t} \PY{o}{*} \PY{n}{np}\PY{o}{.}\PY{n}{sin}\PY{p}{(}\PY{n}{theta}\PY{p}{)} \PY{o}{\PYZhy{}} \PY{p}{(}\PY{l+m+mf}{0.5} \PY{o}{*} \PY{n}{g} \PY{o}{*} \PY{n}{t}\PY{o}{*}\PY{o}{*}\PY{l+m+mi}{2}\PY{p}{)}
          
          \PY{n}{fig} \PY{o}{=} \PY{n}{plt}\PY{o}{.}\PY{n}{figure}\PY{p}{(}\PY{p}{)}
          \PY{n}{ax} \PY{o}{=} \PY{n}{plt}\PY{o}{.}\PY{n}{axes}\PY{p}{(}\PY{n}{xlim}\PY{o}{=}\PY{p}{(}\PY{l+m+mi}{0}\PY{p}{,} \PY{l+m+mi}{10}\PY{p}{)}\PY{p}{,} \PY{n}{ylim}\PY{o}{=}\PY{p}{(}\PY{l+m+mi}{0}\PY{p}{,} \PY{l+m+mi}{5}\PY{p}{)}\PY{p}{)}
          
          \PY{c+c1}{\PYZsh{} \PYZsh{} Object to animate}
          \PY{n}{point}\PY{p}{,} \PY{o}{=} \PY{n}{ax}\PY{o}{.}\PY{n}{plot}\PY{p}{(}\PY{p}{[}\PY{l+m+mi}{1}\PY{p}{]}\PY{p}{,} \PY{p}{[}\PY{l+m+mi}{1}\PY{p}{]}\PY{p}{,} \PY{n}{marker}\PY{o}{=}\PY{l+s+s1}{\PYZsq{}}\PY{l+s+s1}{o}\PY{l+s+s1}{\PYZsq{}}\PY{p}{,} \PY{n}{ms}\PY{o}{=}\PY{l+m+mi}{40}\PY{p}{)}  \PY{c+c1}{\PYZsh{} for points}
          
          \PY{k}{def} \PY{n+nf}{animate}\PY{p}{(}\PY{n}{i}\PY{p}{)}\PY{p}{:}
          
              \PY{n}{i} \PY{o}{/}\PY{o}{=} \PY{l+m+mi}{10}
              \PY{k}{if} \PY{n}{i} \PY{o}{==} \PY{l+m+mi}{0}\PY{p}{:}
                  \PY{n}{s} \PY{o}{=} \PY{n}{i}
              \PY{k}{else}\PY{p}{:}
                  \PY{n}{s} \PY{o}{=} \PY{n}{np}\PY{o}{.}\PY{n}{linspace}\PY{p}{(}\PY{l+m+mi}{0}\PY{p}{,} \PY{n}{i}\PY{p}{)}
              
          
              \PY{n}{x} \PY{o}{=} \PY{n}{v} \PY{o}{*} \PY{n}{s} \PY{o}{*} \PY{n}{np}\PY{o}{.}\PY{n}{cos}\PY{p}{(}\PY{n}{theta}\PY{p}{)}
              \PY{n}{y} \PY{o}{=} \PY{n}{v} \PY{o}{*} \PY{n}{s} \PY{o}{*} \PY{n}{np}\PY{o}{.}\PY{n}{sin}\PY{p}{(}\PY{n}{theta}\PY{p}{)} \PY{o}{\PYZhy{}} \PY{p}{(}\PY{l+m+mf}{0.5} \PY{o}{*} \PY{n}{g} \PY{o}{*} \PY{n}{s}\PY{o}{*}\PY{o}{*}\PY{l+m+mi}{2}\PY{p}{)}
              \PY{n}{ax}\PY{o}{.}\PY{n}{plot}\PY{p}{(}\PY{n}{x}\PY{p}{,} \PY{n}{y}\PY{p}{,} \PY{l+s+s1}{\PYZsq{}}\PY{l+s+s1}{c}\PY{l+s+s1}{\PYZsq{}}\PY{p}{)}
              
              \PY{k}{if} \PY{n}{i} \PY{o}{==} \PY{l+m+mi}{0}\PY{p}{:}
                  \PY{n}{point}\PY{o}{.}\PY{n}{set\PYZus{}data}\PY{p}{(}\PY{n}{x}\PY{p}{,} \PY{n}{y}\PY{p}{)}   
              \PY{k}{else}\PY{p}{:}  
                  \PY{n}{point}\PY{o}{.}\PY{n}{set\PYZus{}data}\PY{p}{(}\PY{n}{x}\PY{p}{[}\PY{o}{\PYZhy{}}\PY{l+m+mi}{1}\PY{p}{]}\PY{p}{,} \PY{n}{y}\PY{p}{[}\PY{o}{\PYZhy{}}\PY{l+m+mi}{1}\PY{p}{]}\PY{p}{)}    
              \PY{k}{return}  \PY{p}{(}\PY{n}{point}\PY{p}{,}\PY{p}{)}
              
          \PY{n}{ani} \PY{o}{=} \PY{n}{animation}\PY{o}{.}\PY{n}{FuncAnimation}\PY{p}{(}\PY{n}{fig}\PY{p}{,} \PY{n}{animate}\PY{p}{,} \PY{n}{frames}\PY{o}{=}\PY{l+m+mi}{15}\PY{p}{,} \PY{n}{interval}\PY{o}{=}\PY{l+m+mi}{50}\PY{p}{)}
          \PY{n}{ani}
\end{Verbatim}


\begin{Verbatim}[commandchars=\\\{\}]
{\color{outcolor}Out[{\color{outcolor}273}]:} <matplotlib.animation.FuncAnimation at 0x181eb248d0>
\end{Verbatim}
            
    \begin{center}
    \adjustimage{max size={0.9\linewidth}{0.9\paperheight}}{output_139_1.png}
    \end{center}
    { \hspace*{\fill} \\}
    
    \section{Summary}\label{summary}

 - \texttt{scipy.integrate.odeint} can be used to find a numerical
approximation for the change in a parameter over a given range of input
values (expressed as an ordinary differential equation, ode). - This
allows us to easily deal with dicontinuous functions. - An can be solved
for a specific \emph{solution} value by interpolating. - We can find the
value of the function at a specific value by selecting the points at
which the function is approximated. - We can create interactive plots
with \texttt{ipywidgets}. - \texttt{matplotlib.animation} and
\texttt{matplotlib.rc} can be used to produce animated plots. -
\texttt{IPython.display.HTML} allows us to view and play these
animations within Jupyter notebook. - Installing the program
\texttt{ffmpeg} allows us to save an animation in a format we can view
outside of the Python environment.


    % Add a bibliography block to the postdoc
    
    
    
    \end{document}
